\documentclass[12pt, a4paper]{article}

\usepackage[slovak]{babel}
\usepackage[utf8]{inputenc}
\usepackage[T1]{fontenc}
\usepackage{geometry}
\usepackage{hyperref}
\usepackage{csquotes}

\usepackage[style=iso-numeric,backend=biber]{biblatex}
\addbibresource{references.bib}
\AtBeginBibliography{\small}

\geometry{
	a4paper,
	margin=2.5cm,
	top=2cm,
	bottom=2cm
}

\begin{document}
\begin{titlepage}
    \hspace{0pt}
    \centering
    \vfill
    \large Princípy informačnej bezpečnosti \\
    \vspace{0.4cm}
    \vspace{1cm}
    \large \textbf{Zvýšenie odolnosti webových aplikácií proti útokom typu DDoS, 
          \\za pomoci horizontálneho škálovania  \\}
    \vspace{2.5cm}
    \normalsize Miroslav Hájek \\[0.2cm]
	Akademický rok: 2020 / 2021 \\[0.1cm]
	Fakulta informatiky a informačných technológií, \\
	Slovenská technická univerzita v Bratislave
    \vfill
\end{titlepage}


\pagenumbering{gobble}
\tableofcontents
\newpage
\pagenumbering{arabic}
\setcounter{page}{1}

\section{Dostupnosť ako bezpečnostný atribút}
Zabezpečenie nepretržitého prístupu k webovým službám je očakávanou a takmer nevyhnutnou požiadavkou pre 
akýkoľvek významnejší informačný systém. Predstavuje neoddeliteľnú súčasť obrazu o spoľahlivosti 
ich prevádzkovateľov pôsobiacich vo virtuálnom priestore internetu. Aspekt dostupnosti sa prejavuje tým,
že údaje sú k dispozícii pre autorizovaných používateľov okamžite a bez nečakaných obmedzení. Odlišná 
interpretácia definuje pojem dostupnosti ako ochranu proti zlomyseľnému zatajovaniu informácií. Spoločným 
menovateľom pre oba tieto pohľady je dôraz na všadeprítomnosť služby v ľubovoľnom čase potreby. Taký stav
je vskutku ideálny, ale je možné sa mu aspoň priblížiť predovšetkým identifikáciou bodov zlyhania alebo 
miest prieniku a ich následným systematickým eliminovaním.

Pre bezpečné nakladanie s informáciami nestačí samotná dostupnosť, ale zároveň je potrebné pri návrhu a 
prevádzke systémov myslieť aj na dôvernosť a integritu údajov. Spolu tvoria tradičný model informačnej 
bezpečnosti označovaný ako tzv. \emph{CIA triáda} (Confidentiality, Integrity, Availablity), ktorý sa často 
spolieha na vyváženosť a rovnocennosť týchto troch prvkov \cite{availability}. Nutno poznamenať, 
že to úplne neplatí, pretože nežiadaným znemožnením prístupu k zdrojom sa ich neporušenosť a zabezpečenosť 
proti neoprávnenému prezeraniu, či úprave, stáva bezpredmetná. O dostupnosť sa teda opierajú všetky ďalšie
bezpečnostné predpoklady, ktoré má systém napĺňať.

\subsection{Kľúčový zabezpečovatelia dostupnosti}
Komunikačné technológie často tvoria chrbtovú kosť väčšiny moderných biznisov pričom ich hlavnou úlohou je 
sprostredkovanie informácií naprieč spoločnosťou a podieľajú sa na riadení podnikových procesov. Okrem 
ľudského kapitálu sa beh digitálnej organizácie spolieha spravidla na tri prvky: \emph{softvér, hardvér a 
počítačovú sieť} \cite{availability}. 

\paragraph{Softvér:}
Softvér je najkritickejším komponentom spomedzi vymenovaných, pretože na základe príkazov v kóde programov 
je ovládaný hardvér a sieťové zariadenia. Všetky potenciálne útoky a ich dopady musia byť riešené primárne 
na úrovni softvéru. Na napadnutie sú využívané zraniteľnosti v ovládacích a regulačných mechanizmoch 
systému, sprístupnené prelomením nedostatočne zabezpečeného verejného rozhrania služieb. Najčastejším cieľom 
útočníka je dosiahnutie kontroly nad zariadením alebo vyvolaním chaosu vo fungovaní prevádzky. 
 
Zlyhanie programového vybavenia nemusí byť iba v dôsledku nepriaznivých vonkajších vplyvov, ale tiež sa 
prejavujú chyby spôsobené nekorektným návrhom alebo implementáciou systému s odchýlkami od
požadovaného správania, majúce nepriaznivý vplyv napríklad na dostupnosť konkrétnej stránky. Tieto chyby sú 
vnesené neúmyselne najčastejšie programátorom. Počas behu aplikácie môžu nastať zlyhania operačného 
prostredia zapríčinené nedostatkom pamäte pri alokácii, zaplneniu diskového úložiska alebo uviaznutím 
systému. 

\paragraph{Hardvér:}
Naproti tomu poruchy hardvéru bývajú zriedkavejšie, ale o to podstatne fatálnejšie pre celkový chod, keď
nie je ich výpadok adresovaný redundanciou komponentov. Prinavrátenie do behu znamená výmenu zariadenia za 
iné prevádzkyschopné, či už dočasnou úpravou fyzickej infraštruktúry alebo neodkladnou montážou náhrady. 
Duplikáciou napr. diskov cez RAID dosiahneme síce vyššiu dostupnosť ale za cenu integrity z dôvodu zdvojením 
dát \cite{availability}, preto je vždy potrebné mať na pamäti vyváženosť bezpečnostných vlastností navzájom.

\paragraph{Sieť:}
Obmedzením počítačových sietí je ich priepustnosť podmienená šírkou prenosového pásma a réžiou 
spotrebovanou na obsluhu zvolených komunikačných protokolov. Problémy nastávajú v situáciach, kedy prichádza
k zahlteniu na sieťovej linke. Pri normálnom chode rešpektujú uzly v sieti signalizáciu zaznačenú do 
paketov a prispôsobia rýchlosť vysielania, čím sa za istý čas uľaví náporu. Útočník, ktorý chce saturovať 
serverové pripojenie to  pochopiteľne nerešpektuje a preto by sa nadmerná premávka mala presmerovať a 
filtrovať. Smerovač určite spôsobí straty paketov a zvýšené oneskorenie vystavený niekoľkonásobnej záťaži 
než podporuje. Okrem spustenia čo najväčšieho toku paketov do siete sa dajú zneužiť synchronizačné 
očakávania komunikácie, konkrétne stavový automat protokolu TCP.

\subsection{Dôvody viktimizácie prevádzkovateľov}
Nedostupnosť služieb máva za následok, buď priame škody v podobe finančných strát alebo vplýva na 
pošramotenie nadobudnutej reputácie u klientov, ktorí si zlyhanie môžu spájať so stratou spoľahlivosti a 
dôveryhodnosti služby. Čím väčší poskytovateľ a používanejšia webová stránka, tým rozsiahlejšie sú 
potenciálne dopady pri neočakávanej vyradení z prevádzky. Zároveň dochádza k adekvátnemu navýšeniu 
zriadenej odolnosti investovanej do predchádzania a aktívnej defenzívy proti útokom.  

Motivácie a dôvody stojace za aktivitami úmyselne narúšajúcimi dostupnosť zvolených obetí v podobe
webových aplikácií sa líšia od prípadu k prípadu, ale dajú sa zhrnúť do nasledujúcich kategórií 
\cite{why-attack} \cite{ddos-attacks}:
\begin{itemize}
\itemsep0em 
\item \emph{Kapitálové zisky} - nadobudnutie finančnej odplaty od objednávateľa alebo nekalé snahy o 
potopenie konkurencie v ekonomickej súťaži predstavuje významný hnací faktor pre útočníkov. 
Hlavným zámerom je lepšie peňažné zabezpečenie sa nehľadiac na prostriedky vynaložené na ten účel. 
Úspešná realizácia vyžaduje značné technické zručnosti, pretože firmám, ktoré sú hlavných cieľom,
ide o veľa. 

Zvykne sa jednať sa o komerčné webstránky alebo služby finančných inštitúcií (HSBC, BTC a 
Ethereum burzy), servery a sieťové zariadenia poskytovateľov webhostingu alebo internetového pripojenia 
(Deutsche Telekom, OVH, Dyn), herné servery (Steam, Blizzard, EA Sports), emailové servery (Eir).
\cite{why-attack}

\item \emph{Pomsta} - ide o prevažne frustrovaných jednotlivcov snažiacich sa o odplatu za vnímanú 
nespravodlivosť, ktorú podľa nich prevádzkovateľ pácha. Počas operácie Payback z roku 2010 bola odplatou
hackerskej skupiny Anonymous za útok na stránky s torentami a pirátskym softvérom na organizácie
na ochranu autorských práv. V decembri zhodili stránky MasterCard, Visa, Paypal a
iných, ktoré vydržiavali donácie organizácii Wikileaks, pretože publikovala prísne tajné informácie
americkej vlády.

\item \emph{Ideologické a politické presvedčenie} - útočník sa snaží dať hlasno najavo svoj nesúhlas s 
ideovo protichodnými názormi a postojmi znefunkčnením alebo poškodením platformy prostredníctvom ktorej 
oponent pôsobí alebo šíri svoj svetonázor. Takáto forma útokov sa označuje tiež za hacktivizmus, ktorého 
prívrženci hájia slobodu slova a právo na súkromie proti nadmernému sledovaniu. Pre predstavu napádaných 
webstránok boli prominentné útoky roku 2016 cielené na skupiny ako Black Lives Matter, Ku Klux Klan, 
Wikileaks, oboch prezidentských kandidátov v USA, taliansku a írsku vládu alebo európsku komisiu. 
\cite{why-attack}

\item \emph{Demonštrácia schopností} - ide o experimentálne útoky so zámerom hackera vyskúšať si nové techniky alebo predviesť svoje kompetencie. 
\item\emph{Kybernetický terorizmus} - útočník je súčasťou vojenskej alebo teroristickej operácie s cieľom
poškodenia nepriateľovi.
\end{itemize}

Výber primeranej obete má taktiež svoj podiel na úspešnosti škodlivého zásahu útočníka do prevádzky,
pretože ten sa výrazne nelíši od konvenčného zločinu. Medzi obvykle spomínané zdôvodnenia páchania 
kriminality zo sociologickej perspektívy patrí kriminologická teória známa pod názvom 
\emph{teória rutinných aktivít}. Poskytuje predpoklady, aké musí daný subjekt spĺňať nato, aby bol 
zasiahnutý. Vyslovuje, že zločin sa udeje vtedy, keď motivovaný útočník, so sklonmi na páchanie trestnej 
činnosti, príde do stretu s objektom ponechaného bez prítomnosti schopného strážcu \cite{cohen-felson}.

Pokiaľ sú vytvorené priaznivé okolnosti na prelomenie do systému, napríklad predvoleným prepustením 
všetkej sieťovej premávky je veľká šanca, že motivovaný útočník sa bude snažiť o zneužitie a jednoduchšie
je mu umožnené zhodiť tento systém. Na druhej stane pri absencii jediného kritéria, sa buď znižujú šance na
viktimizáciu alebo úplne vylúči. Takáto situácia vie odradiť značné množstvo potenciálnych útočníkov.
Podľa \emph{teórie racionálneho jednania} koná útočník pri zvažovaní realizácie svojho činu racionálne,
hoc sa jedná o obmedzenú rozumnosť a síce z uhľa pohľadu delikventa ide o cieľavedomé rozhodnutie, kde 
pozitívny obnos prevažuje nad možnými rizikami uskutočnenia.

Vhodnosť zamerania sa na zvolenú infraštruktúru pre potenciálneho páchateľa je zachytiteľné kritériami 
\emph{VIVA} (Value, Inetria, Visibility, Accesibility) \cite{why-attack}. V kontexte útokov odmietnutia 
služby sa pod \emph{hodnotou} rozumie dôležitosť rozbitia cieľa pre útočníka, presnejšie či daný internetový 
portál alebo herný server dosahuje dostatočné zisky, aby ich odstavením spôsobil dostatočnú škodu. 
\emph{Zotrvačnosťou} sa myslí odpor, ktorý kladie infraštruktúra voči útoku rozličnými bezpečnostnými 
mechanizmami. Medzi priamočiare praktiky patrí udržiavanie aktualizovanému systému so zaplátanými 
zraniteľnosťami alebo obmedzením počtu dopytov z jednej adresy. Služby s veľkou zotrvačnosťou sú schopné 
ustáť väčší nápor v prípade napadnutia. \emph{Viditeľnosť} predstavuje rozsah v akom je webstránka verejne 
dostupná a známa širšiemu publiku. \emph{Prístupnosť} značí jednoduchosť v dosiahnutí vytýčeného sieťových 
uzlov, ktoré majú predstavovať obeť, použitou taktikou útoku bez povšimnutia. Rovnako sa spája so 
schopnosťou nezanechať stopy na mieste činu následkom neprítomnosti mechanizmu monitorovania a detekcie 
narušenia. 

Relatívne vysokou hodnotou, viditeľnosťou a prístupnosťou a nízkou zotrvačnosťou sa cieľ stáva 
exponovanejší a tým žiadanejší pre útočníka na odstavenie. Oproti klasickému zločinu, kedy býva nutná 
prítomnosť páchateľa a obeta na jednom mieste, kyberzločin dovoľuje útočníkovi pôsobiť cez internet takmer 
od hocikadiaľ a maskovať sa proti odhaleniu pomocou prostredníkov ako sú na pohľad neškodné servery na 
komunikáciu s infikovanými zariadeniami alebo anonymizačnými protokolmi, napríklad Tor.

Pokiaľ by nejestvovali indivíduá so zámerom druhému spôsobiť ujmu vyplývajúcu z nastolených motivačných
faktorov nebolo by ani potrebné sa výrazne zaoberať zvyšovaním odolnosti informačných služieb, či brániť
voči kriminalite ako takej. Určité vzorce ľudského správania naznačujú, že je prakticky nemožné sa pred 
týmito spoločenskými javmi vyhraniť. Človek má tendenciu sa asimilovať v sociálnej skupine a preberať prvky 
jej správania \cite{heretik}. Základné sociálne jednotky ako rodina, vrstovníci alebo osoby rovnakého
statusu v spoločnosti môžu mať pozitívny vplyv na inhibíciu antisociálneho správania, či naopak majú
za následok normalizáciu všeobecnejšie neakceptovaných noriem v dôsledku sociálneho tlaku skupiny.
Naopak, prílišné dovoľujúce prostredie počas personálneho vývinu má za potenciálny následok osvojenie si 
negatívnych vzorov správania, pretože chýbajú podmieňujúce zážitky, ktoré by korektne usmernili ponímanie 
vážnosti nesprávnych činov. Vyrastanie v neláskyplnom prostredí s prejavmi menejcennosti môže 
neskôr viesť k tomu, že delikventov teší širší záujem o ich záporné činy, tým že boli konečne povšimnutý a 
uznaný vo svojej aktivite. Vyjadrujú tak nadobudnutú nevraživosť a konajú na protest proti 
reštriktívnym orgánom verejnej moci a ponímanej nespravodlivosti.

Teória diferenciálnej asociácie tvrdí, že v spoločnosti existujú paralelne tak 
prosociálne, ako aj asociálne normy, postoje a spôsoby správania \cite{heretik}, čím sa vysvetľujú 
trestné činy dostatočne zabezpečených jedincov strednej vrstvy. Dochádza u nich k stotožňovaniu sa s 
antisociálnymi prístupmi na ceste za osobným úspechom. Zároveň páchatelia podvedome zľahčujú následky 
sociálneho zlyhania v spojitosti s stanovenými antikriminálnymi normami. V snahu neutralizovať svoje
konanie popierajú zodpovednosti skrývaním sa za bezvýchodiskovosť situácie, neuznávajú význam obete 
prenášajúc naň vinu a ohraďujú sa konaním v záujme vyššieho princípu. Na základe téorie etiketovania
je delikvencia len momentálny stav osobnosti, ktorý pramálo súvisí s psychickými vlastnosťami a správaním.
Z množstva sociálno-psychologických postulátov je zrejmé, že separátne nedokážu jednoznačne predpovedať 
exaktnú profiláciu hľadaného delikventa, ale poskytujú komplexnejší obraz o tom, aké pohnútky ich môžu viesť
k zdanlivo neočakávaného činu na konkrétnej obeti.

Predpokladom na spáchanie kybernetického zločinu sústrediac sa na snahu o eliminovanie dostupnosti
sú okrem úvodných pohnútok aj určité technické zručnosti útočníka. Keď chce byť pracovník slobodný od 
nadriadených a má, hoc aj nedopatrením, záznam v registri trestov je prinútený aj z dôvodu vlastnej 
otáznejšej zamestnateľnosti uchýliť sa niekedy k predávaniu alebo prenajímaniu strojov, ktoré sa mu podarí 
kompromitovať, pre niekoho kto má záujem učiniť veľkoplošný útok a vedú ho možno dôvody uvedenej skorej.
Časom to hacker berie ako formu biznisu, síce ilegálneho, ale dokáže mu zarobiť uspokojivý obnos 
\cite{infiltrating-botnet}. Využívané programy sú vymieňané alebo predávané sprostredkovane cez fóra,
sú reklamované a tamojšími administrátormi overované podobne ako komerčný softvér. Na fórach prebiehajú
diskusie a návody k schodným spôsobom ako si dostupný malvér sprevádzkovať a upraviť podľa potreby.

Vykonanie útoku so zámerom obmedziť dostupnosť cudzieho systém je podľa platnej legislatívnej úpravy 
na Slovensku trestným činom podľa \emph{§247a Trestného zákona} a obdoby sú zavedené v právnych poriadkov aj
iných krajín (napr. v USA - Computer Fraud and Abuse Act a 18 U.S.C. § 1030, v Nemecku - Strafgesetzbuch: 
§303b  Computersabotage). Činnosť neoprávneného zásahu do počítačového systému spadá v podstate do rovnakej 
oblasti ako poškodenie cudzieho majetku s trestami odňatia slobody pri preukázaní, od šiestich mesiacov až 
po 10 rokov podľa závažnosti\cite{trestny-zakon}. 

\section{Anatómia útokov Denial of Service}



4-7str.
IRC, Botnet a Taxonómia útokov, Slide 3

Ako sa majú potenciálne obete chrániť – Viacvrstvové fyzické zabezpečenie pevností ako paralela k virtuálnemu (Steny, Brány, Zámky a dvere, Svetlá) – Viacero línií obrany


 nasledujú peniaze od veľkých firiem a čo sa im oplatí (DNS závislé), botnet ako 
produkt (zdroj 3)

Na zamedzenie 
podobných praktík sa vo všeobecnosti zvykne uplatňovať metóda viacerých línií obrany, ktoré sa osvedčili už 
za dávnych čias pri fyzickej obrane pevností. Pri vtedy praktikovaných technikách sú patrné mnohé podobnosti 
so súčasnou ochranou webových aplikácií.

Bezpečné prostredie má ľudí a aktíva chrániť pred ujmou a stratou. Staroveké civilizácie si na ten účel 
stavali obranné pevnosti, ktoré mali poskytnúť ochranu pre veci vnútri nich. Už Mykénska civilizácia 
uplatňovala štyri zóny fyzickej obrany pred nepriateľom \cite{physical-security}. Prvou líniou obrany
bola priekopa alebo rieka, za ktorou sa v druhej línii nachádzala obvodový múr. Menšia vnútorná stena
obkolesovala svätostánky a palác. Posledná najvnútornejšia obranná stena stála pred pokladom v kráľovskej 
štvrti. 

Blackhole je priekopa, Firewall je brána s ACL
Prostredie Internetu a prvky jeho návrhu, ktoré nenahrávahú bezpečnosti ako baked-in konceptu
Ako sa majú potenciálne obete chrániť – Viacvrstvové fyzické zabezpečenie pevností ako paralela k virtuálnemu (Steny, Brány, Zámky a dvere, Svetlá) – Vicerero línií obrany
Hra na mačku a myš a teória bigger garden hose (spomenutie aké ujmy spôsobili vybrané útoky)
Architektúra  a opis botnetov (ich vznik história – hra na prebratie kanálov) – agent-handler, IRC , web based – Mirai analysis
Fázy útoku a opis použitých techník na ich realizáciu (automatické, semi-auto, manuálne)
- širi sa ako červ, skrýva sa podobne ako vírusy
- nábor / verbovanie – techniky skenovania potenciálnych hostiteľov a zranitelností 
-  obsadenie (exploit) a nakazenie – typy rozširovania nákazy (centralized, back-chaining, auto)
-  koordinácia útoku a typy útokov
	podvrhnutie ip adresy
	dynamika útoku
	filtrovateľné/nefiltrovateľné
	- Net bandwitdh (TCP protokolové zraniteľnosti SYN, UDP, ICMP flood)
	- App attack (HTTP flood, DrDoS – Distributed reflector attack -DNS, NTP)
- Štatistiky obľúbených útokov za 2019 od ENISA
- typy obrany –
	preventívna (záplaty, firewall, IDS, Resource accounting/multiplication)
	reaktívna (Pattern vs Anonomaly detection)

\section{Škálovanie webových aplikácií}
7 - 13str.
Definuj verikálne a horizontálne škálovanie
IEEE802.3az, LACP - Trunking a konfigurácia na cisco switch
RFC2991 - ECMP smerovanie v OSPFv2 a výber vhodných next hops a problémy s rôznymi trasami
router ospf 1
maximum-paths 2
RFC2827 – Network Ingress filtering – Unicast Reverse Path Forwarding Strict/Loose a FIB (CAM)
cisco sonfig 
RTBH – Remotely triggered black hole with iBGP peer trigger
RFC 3022 – NAT a firewall – routing, load balancing ( Network Address Port Translation(NAPT)) – Popis vytvorenia spojenia v stavovej tabuľke a ak nemá unbound asociáciu tak je zahodený.

- attack response strategy – agent identif., rate limiting, filtering (ingress/egreess/black-hole), reconfig of toplogy
Reverzná proxy – pool serverov, DNS A záznamy (ukážka BIND konfigurácie bez PTR)

\subsection{Agregácia liniek}

\subsection{Smerovanie zamedzujúce nadmernej sieťovej premávke}

\subsection{Algoritmy vyvažovania záťaže}

\subsection{Vysoká dostupnosť služieb}
Opísať rozdiel medzi základnou dostupnosťou a vysokou dostupnosťou – zdroj 1
Protokol VRRP – mechanizmus fungovania, virtálna ip adresa – paket traces
Keepalived konfigurácia dvoch Linux zariadení s failover

\subsection{Softvérové nástroje na vyvažovanie záťaže}

\subsubsection{HAProxy}

\subsubsection{Nginx}

\subsubsection{Dispatcher Worker model}


\section{Ochrana webových serverov}

\subsection{Hlavičky protokolu HTTP}

\subsection{Access Control List}

\subsection{TLS a X.509 certifikáty}



\section{Monitorovanie webovej aplikácie}

\subsection{Metriky}

\subsection{Syslog a Common Logfile Format}

\subsection{Zabbix}

\section{Simulácie útokov a záťažové testy}
Docker-compose: útoky ---> Haproxy/Nginx ---> 1-4x Apache Web site

Typy útokov (UDP flood, SYN Flood, Slow Loris so spoofed IP)
Meniť počet horizontálych inštancií
Meniť počet workerov

Pridaj do zabbixu ostatné hosty
Monitoring na 
	- Apache (Access Log, Error log)
	- Haproxy (HAlog, https://www.haproxy.com/blog/introduction-to-haproxy-logging/)
	- NGINX z logov (https://docs.nginx.com/nginx/admin-guide/monitoring/logging/)
Sleduj network traffic na eth0 pri záplave:  vnstat, zabbix
Sleduj počet spojení pri load balancingu


Dnes - dokončenie 1. sekcie do 14:00
a úvod do DDoS útokov + PSI
Sobota - PSI, dokončiť 2 sekciu
Nedeľa - polka tretej sekcie, odstovať záťažové testy a zozbierať logy z apache, haproxy a nginx
po ab, slowhttptest
logy z vnstat a wireshark po hping flood (udp, icmp, syn) a vlastnom skripte - aj pc server
\begin{verbatim}
ab -n 100 -c 10 http://192.168.100.82/

hping 192.168.100.82 --udp -p 80 --flood  // hping 192.168.100.82 --syn -p 8090 -a 192.168.1.1

slowhttptest -c 1000 -H -g -o my_header_stats -i 10 -r 200 -t GET -u https://192.168.100.82/ -x 24 -p 10
thc-ssl-dos 192.168.100.82 80 --accept

wireshark
\end{verbatim}

\printbibliography[title={Literatúra}]

\end{document}
