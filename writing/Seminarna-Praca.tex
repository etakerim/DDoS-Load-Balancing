\documentclass[11pt, a4paper]{article}

\usepackage[slovak]{babel}
\usepackage[utf8]{inputenc}
\usepackage[T1]{fontenc}
\usepackage{geometry}
\usepackage{hyperref}
\usepackage{csquotes}

\usepackage[style=iso-authoryear,backend=biber]{biblatex}
\addbibresource{references.bib}
\AtBeginBibliography{\small}

\geometry{
	a4paper,
	margin=2.5cm,
	top=2cm,
	bottom=2cm
}

\begin{document}
\begin{titlepage}
    \hspace{0pt}
    \centering
    \vfill
    \large Princípy informačnej bezpečnosti \\
    \vspace{0.4cm}
    \vspace{1cm}
    \Large \textbf{ Zvýšenie odolnosti webových aplikácií proti útokom typu DDoS, 
          \\za pomoci horizontálneho škálovania  \\}
    \vspace{2.5cm}
    \normalsize Miroslav Hájek \\[0.2cm]
	Akademický rok: 2020 / 2021 \\[0.1cm]
	Fakulta informatiky a informačných technológií, \\
	Slovenská technická univerzita v Bratislave
    \vfill
\end{titlepage}



\pagenumbering{gobble}
\tableofcontents
\pagenumbering{arabic}
\setcounter{page}{1}

\section{Dostupnosť ako bezpečnostný atribút}


\section{Anatómia útokov Denial of Service}


\section{Škálovanie webových aplikácií}

\subsection{Agregácia liniek}

\subsection{Smerovanie zamedzujúce nadmernej sieťovej premávke}

\subsection{Algoritmy vyvažovania záťaže}

\subsection{Vysoká dostupnosť služieb}

\subsection{Softvérové nástroje na vyvažovanie záťaže}

\subsubsection{HAProxy}

\subsubsection{Nginx}

\subsubsection{Dispatcher Worker model}


\section{Ochrana webových serverov}

\subsection{Hlavičky protokolu HTTP}

\subsection{Access Control List}

\subsection{TLS a X.509 certifikáty}



\section{Monitorovanie webovej aplikácie}

\subsection{Metriky}

\subsection{Syslog a Common Logfile Format}

\subsection{Zabbix}

\section{Simulácie útokov a záťažové testy}


\printbibliography[title={Literatúra}]

\end{document}
