\documentclass[12pt, a4paper]{article}

\usepackage[slovak]{babel}
\usepackage[utf8]{inputenc}
\usepackage[T1]{fontenc}
\usepackage{geometry}
\usepackage{hyperref}
\usepackage{csquotes}

\usepackage[style=iso-numeric,backend=biber]{biblatex}
\addbibresource{references.bib}
\AtBeginBibliography{\small}

\geometry{
	a4paper,
	margin=2.5cm,
	top=2cm,
	bottom=2cm
}

\begin{document}
\begin{titlepage}
    \hspace{0pt}
    \centering
    \vfill
    \large Princípy informačnej bezpečnosti \\
    \vspace{0.4cm}
    \vspace{1cm}
    \large \textbf{Zvýšenie odolnosti webových aplikácií proti útokom typu DDoS, 
          \\za pomoci horizontálneho škálovania  \\}
    \vspace{2.5cm}
    \normalsize Miroslav Hájek \\[0.2cm]
	Akademický rok: 2020 / 2021 \\[0.1cm]
	Fakulta informatiky a informačných technológií, \\
	Slovenská technická univerzita v Bratislave
    \vfill
\end{titlepage}


\pagenumbering{gobble}
\tableofcontents
\newpage
\pagenumbering{arabic}
\setcounter{page}{1}

\section{Dostupnosť ako bezpečnostný atribút}
Zabezpečenie nepretržitého prístupu k webovým službám je očakávanou a takmer nevyhnutnou požiadavkou pre 
akýkoľvek významnejší informačný systém. Predstavuje neoddeliteľnú súčasť obrazu o spoľahlivosti 
ich prevádzkovateľov pôsobiacich vo virtuálnom priestore internetu. Aspekt dostupnosti sa prejavuje tým,
že údaje sú k dispozícii pre autorizovaných používateľov okamžite a bez nečakaných obmedzení. Odlišná 
interpretácia definuje pojem dostupnosti ako ochranu proti zlomyseľnému zatajovaniu informácií. Spoločným 
menovateľom pre oba tieto pohľady je dôraz na všadeprítomnosť služby v ľubovoľnom čase potreby. Taký stav
je vskutku ideálny, ale je možné sa mu aspoň priblížiť predovšetkým identifikáciou bodov zlyhania alebo 
miest prieniku a ich následným systematickým eliminovaním.

Pre bezpečné nakladanie s informáciami nestačí samotná dostupnosť, ale zároveň je potrebné pri návrhu a 
prevádzke systémov myslieť aj na dôvernosť a integritu údajov. Spolu tvoria tradičný model informačnej 
bezpečnosti označovaný ako tzv. \emph{CIA triáda} (Confidentiality, Integrity, Availablity), ktorý sa často 
spolieha na vyváženosť a rovnocennosť týchto troch prvkov \cite{availability}. Nutno poznamenať, 
že to úplne neplatí, pretože nežiadaným znemožnením prístupu k zdrojom sa ich neporušenosť a zabezpečenosť 
proti neoprávnenému prezeraniu, či úprave, stáva bezpredmetná. O dostupnosť sa teda opierajú všetky ďalšie
bezpečnostné predpoklady, ktoré má systém napĺňať.

\subsection{Kľúčový zabezpečovatelia dostupnosti}
Komunikačné technológie často tvoria chrbtovú kosť väčšiny moderných biznisov pričom ich hlavnou úlohou je 
sprostredkovanie informácií naprieč spoločnosťou a podieľajú sa na riadení podnikových procesov. Okrem 
ľudského kapitálu sa beh digitálnej organizácie spolieha spravidla na tri prvky: \emph{softvér, hardvér a 
počítačovú sieť} \cite{availability}. 

\paragraph{Softvér:}
Softvér je najkritickejším komponentom spomedzi vymenovaných, pretože na základe príkazov v kóde programov 
je ovládaný hardvér a sieťové zariadenia. Všetky potenciálne útoky a ich dopady musia byť riešené primárne 
na úrovni softvéru. Na napadnutie sú využívané zraniteľnosti v ovládacích a regulačných mechanizmoch 
systému, sprístupnené prelomením nedostatočne zabezpečeného verejného rozhrania služieb. Najčastejším cieľom 
útočníka je dosiahnutie kontroly nad zariadením alebo vyvolaním chaosu vo fungovaní prevádzky. 
 
Zlyhanie programového vybavenia nemusí byť iba v dôsledku nepriaznivých vonkajších vplyvov, ale tiež sa 
prejavujú chyby spôsobené nekorektným návrhom alebo implementáciou systému s odchýlkami od
požadovaného správania, majúce nepriaznivý vplyv napríklad na dostupnosť konkrétnej stránky. Tieto chyby sú 
vnesené neúmyselne najčastejšie programátorom. Počas behu aplikácie môžu nastať zlyhania operačného 
prostredia zapríčinené nedostatkom pamäte pri alokácii, zaplneniu diskového úložiska alebo uviaznutím 
systému. 

\paragraph{Hardvér:}
Naproti tomu poruchy hardvéru bývajú zriedkavejšie, ale o to podstatne fatálnejšie pre celkový chod, keď
nie je ich výpadok adresovaný redundanciou komponentov. Prinavrátenie do behu znamená výmenu zariadenia za 
iné prevádzkyschopné, či už dočasnou úpravou fyzickej infraštruktúry alebo neodkladnou montážou náhrady. 
Duplikáciou napr. diskov cez RAID dosiahneme síce vyššiu dostupnosť ale za cenu integrity z dôvodu zdvojením 
dát \cite{availability}, preto je vždy potrebné mať na pamäti vyváženosť bezpečnostných vlastností navzájom.

\paragraph{Sieť:}
Obmedzením počítačových sietí je ich priepustnosť podmienená šírkou prenosového pásma a réžiou 
spotrebovanou na obsluhu zvolených komunikačných protokolov. Problémy nastávajú v situáciach, kedy prichádza
k zahlteniu na sieťovej linke. Pri normálnom chode rešpektujú uzly v sieti signalizáciu zaznačenú do 
paketov a prispôsobia rýchlosť vysielania, čím sa za istý čas uľaví náporu. Útočník, ktorý chce saturovať 
serverové pripojenie to  pochopiteľne nerešpektuje a preto by sa nadmerná premávka mala presmerovať a 
filtrovať. Smerovač podporujúci prenosové rýchlosti do 100 Mbit/s určite spôsobí straty paketov a zvýšené 
oneskorenie pri záťaži 1 Gbit/s Okrem spustenia čo najväčšieho toku paketov do siete sa dajú zneužiť napr. 
synchronizačné očakávania komunikácie, konkrétne stavový automat protokolu TCP.

\subsection{Dôvody viktimizácie prevádzkovateľov}
Nedostupnosť služieb máva za následok, buď priame škody v podobe finančných strát alebo vplýva na 
pošramotenie nadobudnutej reputácie u klientov, ktorý si zlyhanie môže spájať so stratou spoľahlivosťou a 
dôveryhodnosti služby. Čím väčší poskytovateľ a používanejšia webová stránka, tým rozsiahlejšie sú 
potenciálne dopady pri vyradení z prevádzky. Zároveň dochádza k adekvátnemu navýšeniu zriadenej odolnosti 
investovanej do predchádzania a aktívnej defenzívy proti útokom.  

Motivácie a dôvody stojace za aktivitami úmyselne narúšajúcimi dostupnosť zvolených obetí v podobe
webových aplikácií sa líšia od prípadu k prípadu, ale rámcovo je ich možné zhrnúť do nasledujúcich
skupín \cite{why-attack} \cite{ddos-attacks}:
\begin{itemize}
\itemsep0em 
\item \emph{Kapitálové zisky} - nadobudnutie finančnej odplaty od objednávateľa alebo nekalé snahy o 
potopenie konkurencie v ekonomickej súťaži predstavuje hnací faktor pre útočníkov. Hlavným zámerom je
lepšie peňažné zabezpečenie sa nehľadiac na prostriedky vynaložené na ten účel. Úspešná realizácia vyžaduje 
značné technické zručnosti, pretože firmám, ktoré sú hlavných cieľom ide o veľa.
\item \emph{Pomsta} - prevažne frustrovaný jednotlivci sa snažia o odplatu za vnímanú nespravodlivosť, 
ktorú podľa nich prevádzkovateľ pácha.
\item \emph{Ideologické a politické presvedčenie} - útočník sa snaží dať hlasno najavo svoj nesúhlas s 
ideovo protichodnými názormi a postojmi znefunkčnením alebo poškodením platformy prostredníctvom ktorej 
oponent pôsobí alebo šíri svoj svetonázor. V súčastnosti je to jeden z najčastejších motívov za útokmi na 
zneprístupnenie služby. Takáto forma útokov sa označuje tiež za hacktivizmus, ktorého prívrženci hájia 
slobodu slova a právo na súkromie proti nadmernému sledovaniu. 
\item \emph{Demonštrácia schopností} - ide o experimentálne útoky so zámerom vyskúšať si nové techniky
alebo predviesť svoje kompetencie. 
\item\emph{Kybernetický terorizmus} - útočník je súčasťou vojenskej alebo teroristickej operácie s cieľom
poškodenia nepriateľovi.
\end{itemize} 

Výber primeranej obete má taktiež svoj podiel na úspešnosti škodlivého zásahu útočníka do prevádzky,
pretože ten sa výrazne nelíši od konvenčného zločinu. Medzi obvykle spomínané zdôvodnenia páchania 
kriminality zo sociologickej perspektívy patrí kriminologická teória známa pod názvom 
\emph{teória rutinných aktivít}. Poskytuje predpoklady, aké  musí daný subjekt spĺňať nato, aby bol 
zasiahnutý. Vyslovuje, že zločin sa udeje vtedy, keď motivovaný útočník, so sklonmi na páchanie trestnej 
činnosti, príde do stretu s objektom ponechaného bez prítomnosti schopného strážcu \cite{cohen-felson}.
Pokiaľ sú vytvorené priaznivé okolnosti na prelomenie sa do systému, napríklad predvoleným prepustením všetkej sieťovej premávky je veľká šanca, že motivovaný útočník sa bude snažiť o zneužitie a jednoduchšie
je mu umožnené zhodiť tento systém. Na druhej stane pri absencii jediné kritéria, buď znižuje šance na
viktimizáciu alebo ich úplne vylúči.

Podľa \emph{teórie racionálneho jednania} koná pritom útočník racionálne, hoc sa jedná o obmedzenú 
rozumnosť a síce z uhľa pohľadu delikventa ide o cieľavedomé rozhodnutie, kde pozitívny obnos prevažuje
nad možnými rizikami uskutočnenia. Opačná situácia vie odradiť značné množstvo potenciálnych útočníkov.

Vhodnosť zamerania sa na zvolenú infraštruktúru pre potenciálneho páchateľa vyjadrujú kritéria \emph{VIVA}.
(Value, Inetria, Visibility, Accesibility) \cite{why-attack}. V kontexte útokov odmietnutia služby sa pod 
\emph{hodnotou} rozumie dôležitosť rozbitia cieľa pre útočníka, presnejšie či daný internetový portál alebo 
herný server dosahuje dostatočné zisky, aby ich odstavením spôsobil dostatočnú škodu. \emph{Zotrvačnosťou} 
sa myslí odpor infraštruktúry voči útoku. Služby s veľkou zotrvačnosťou sú schopné ustáť väčší nápor v 
prípade napadnutia. \emph{Viditeľnosť} predstavuje rozsah v akom je webstránka verejne dostupná a známa 
širšiemu publiku. \emph{Prístupnosť} značí jednoduchosť v dosiahnutí vytýčeného sieťových uzlov, ktoré majú 
predstavovať obeť, použitou taktikou útoku bez povšimnutia. Rovnako tak sa spája so schopnosťou nezanechať 
stopy na mieste činu následkom neprítomnosti mechanizmu monitorovania a detekcie narušenia.


Heretik - profilácia útočníka\cite{heretik}
Aké sú dôsledky postihy za ich konanie podľa zákona?\cite{trestny-zakon}

Z why-attck pridaj do motivácií časté tipy obetí podľa kategórie.\cite{why-attack}

nasledujú peniaze od veľkých firiem a čo sa im oplatí (DNS závislé), botnet ako produkt





\section{Anatómia útokov Denial of Service}
2-3str.
IRC, Botnet a Taxonómia útokov, Slide 3

Ako sa majú potenciálne obete chrániť – Viacvrstvové fyzické zabezpečenie pevností ako paralela k virtuálnemu (Steny, Brány, Zámky a dvere, Svetlá) – Viacero línií obrany

Na zamedzenie 
podobných praktík sa vo všeobecnosti zvykne uplatňovať metóda viacerých línií obrany, ktoré sa osvedčili už 
za dávnych čias pri fyzickej obrane pevností. Pri vtedy praktikovaných technikách sú patrné mnohé podobnosti 
so súčasnou ochranou webových aplikácií.

Bezpečné prostredie má ľudí a aktíva chrániť pred ujmou a stratou. Staroveké civilizácie si na ten účel 
stavali obranné pevnosti, ktoré mali poskytnúť ochranu pre veci vnútri nich. Už Mykénska civilizácia 
uplatňovala štyri zóny fyzickej obrany pred nepriateľom \cite{physical-security}. Prvou líniou obrany
bola priekopa alebo rieka, za ktorou sa v druhej línii nachádzala obvodový múr. Menšia vnútorná stena
obkolesovala svätostánky a palác. Posledná najvnútornejšia obranná stena stála pred pokladom v kráľovskej 
štvrti. 

Blackhole je priekopa, Firewall je brána s ACL
Prostredie Internetu a prvky jeho návrhu, ktoré nenahrávahú bezpečnosti ako baked-in konceptu
Ako sa majú potenciálne obete chrániť – Viacvrstvové fyzické zabezpečenie pevností ako paralela k virtuálnemu (Steny, Brány, Zámky a dvere, Svetlá) – Vicerero línií obrany
Hra na mačku a myš a teória bigger garden hose (spomenutie aké ujmy spôsobili vybrané útoky)
Architektúra  a opis botnetov (ich vznik história – hra na prebratie kanálov) – agent-handler, IRC , web based – Mirai analysis
Fázy útoku a opis použitých techník na ich realizáciu (automatické, semi-auto, manuálne)
- širi sa ako červ, skrýva sa podobne ako vírusy
- nábor / verbovanie – techniky skenovania potenciálnych hostiteľov a zranitelností 
-  obsadenie (exploit) a nakazenie – typy rozširovania nákazy (centralized, back-chaining, auto)
-  koordinácia útoku a typy útokov
	podvrhnutie ip adresy
	dynamika útoku
	filtrovateľné/nefiltrovateľné
	- Net bandwitdh (TCP protokolové zraniteľnosti SYN, UDP, ICMP flood)
	- App attack (HTTP flood, DrDoS – Distributed reflector attack -DNS, NTP)
- Štatistiky obľúbených útokov za 2019 od ENISA
- typy obrany –
	preventívna (záplaty, firewall, IDS, Resource accounting/multiplication)
	reaktívna (Pattern vs Anonomaly detection)

\section{Škálovanie webových aplikácií}
Definuj verikálne a horizontálne škálovanie
IEEE802.3az, LACP - Trunking a konfigurácia na cisco switch
RFC2991 - ECMP smerovanie v OSPFv2 a výber vhodných next hops a problémy s rôznymi trasami
router ospf 1
maximum-paths 2
RFC2827 – Network Ingress filtering – Unicast Reverse Path Forwarding Strict/Loose a FIB (CAM)
cisco sonfig 
RTBH – Remotely triggered black hole with iBGP peer trigger
RFC 3022 – NAT a firewall – routing, load balancing ( Network Address Port Translation(NAPT)) – Popis vytvorenia spojenia v stavovej tabuľke a ak nemá unbound asociáciu tak je zahodený.

- attack response strategy – agent identif., rate limiting, filtering (ingress/egreess/black-hole), reconfig of toplogy
Reverzná proxy – pool serverov, DNS A záznamy (ukážka BIND konfigurácie bez PTR)

\subsection{Agregácia liniek}

\subsection{Smerovanie zamedzujúce nadmernej sieťovej premávke}

\subsection{Algoritmy vyvažovania záťaže}

\subsection{Vysoká dostupnosť služieb}
Opísať rozdiel medzi základnou dostupnosťou a vysokou dostupnosťou – zdroj 1
Protokol VRRP – mechanizmus fungovania, virtálna ip adresa – paket traces
Keepalived konfigurácia dvoch Linux zariadení s failover

\subsection{Softvérové nástroje na vyvažovanie záťaže}

\subsubsection{HAProxy}

\subsubsection{Nginx}

\subsubsection{Dispatcher Worker model}


\section{Ochrana webových serverov}

\subsection{Hlavičky protokolu HTTP}

\subsection{Access Control List}

\subsection{TLS a X.509 certifikáty}



\section{Monitorovanie webovej aplikácie}

\subsection{Metriky}

\subsection{Syslog a Common Logfile Format}

\subsection{Zabbix}

\section{Simulácie útokov a záťažové testy}
Docker-compose: útoky ---> Haproxy/Nginx ---> 1-4x Apache Web site
iftop, iperf -s, iperf -i 1 -c 192.168.1.99

Typy útokov (UDP flood, SYN Flood, Slow Loris so spoofed IP)
Meniť počet horizontálych inštancií
Meniť počet workerov

Pridaj do zabbixu ostatné hosty – monitoring na Apache a Haproxy,
Sleduj network traffic na eth0 pri záplave
Sleduj počet spojení pri load balancingu
\begin{verbatim}
ab -n 100 -c 10 http://192.168.100.82/

t50 192.168.100.82 -S
hping 192.168.100.82 --udp -p 80 --flood  // hping 192.168.100.82 --syn -p 8090 -a 192.168.1.1

slowhttptest -c 1000 -H -g -o my_header_stats -i 10 -r 200 -t GET -u https://192.168.100.82/ -x 24 -p 10
thc-ssl-dos 192.168.100.82 80 --accept

wireshark
\end{verbatim}

\printbibliography[title={Literatúra}]

\end{document}
