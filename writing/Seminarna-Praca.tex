\documentclass[12pt, a4paper]{article}

\usepackage[slovak]{babel}
\usepackage[utf8]{inputenc}
\usepackage[T1]{fontenc}
\usepackage{geometry}
\usepackage{hyperref}
\usepackage{csquotes}
\usepackage{listings}
\usepackage{setspace}
\setstretch{1.25}

\usepackage[style=iso-numeric,backend=biber]{biblatex}
\addbibresource{references.bib}
\AtBeginBibliography{\small}
\lstset{
	language=bash,
	basicstyle=\ttfamily\footnotesize, 
	extendedchars=true, 
	texcl=true,
	breaklines=true
}

\geometry{
	a4paper,
	margin=2.5cm,
	top=2.5cm,
	bottom=2.5cm
}

\lstset{inputencoding=utf8,texcl=true}

\begin{document}
\begin{titlepage}
    \hspace{0pt}
    \centering
    \vfill
    \large Princípy informačnej bezpečnosti \\
    \vspace{0.4cm}
    \vspace{1cm}
    \large \textbf{Zvýšenie odolnosti webových aplikácií proti útokom typu DDoS, 
          \\za pomoci horizontálneho škálovania  \\}
    \vspace{2.5cm}
    \normalsize Miroslav Hájek \\[0.2cm]
	Akademický rok: 2020 / 2021 \\[0.1cm]
	Fakulta informatiky a informačných technológií, \\
	Slovenská technická univerzita v Bratislave
    \vfill
\end{titlepage}


\pagenumbering{gobble}
\tableofcontents
\newpage
\pagenumbering{arabic}
\setcounter{page}{1}

\section{Dostupnosť ako bezpečnostný atribút}
Zabezpečenie nepretržitého prístupu k webovým službám je očakávanou a takmer nevyhnutnou požiadavkou pre 
akýkoľvek významnejší informačný systém. Predstavuje neoddeliteľnú súčasť obrazu o spoľahlivosti 
ich prevádzkovateľov pôsobiacich vo virtuálnom priestore internetu. Aspekt dostupnosti sa prejavuje tým,
že údaje sú k dispozícii pre autorizovaných používateľov okamžite a bez nečakaných obmedzení. Odlišná 
interpretácia definuje pojem dostupnosti ako ochranu proti zlomyseľnému zatajovaniu informácií. Spoločným 
menovateľom pre oba tieto pohľady je dôraz na všadeprítomnosť služby v ľubovoľnom čase potreby. Taký stav
je vskutku ideálny, ale je možné sa mu aspoň priblížiť predovšetkým identifikáciou bodov zlyhania alebo 
miest prieniku a ich následným systematickým eliminovaním.

Pre bezpečné nakladanie s informáciami nestačí samotná dostupnosť, ale zároveň je potrebné pri návrhu a 
prevádzke systémov myslieť aj na dôvernosť a integritu údajov. Spolu tvoria tradičný model informačnej 
bezpečnosti označovaný ako tzv. \emph{CIA triáda} (Confidentiality, Integrity, Availablity), ktorý sa často 
spolieha na vyváženosť a rovnocennosť týchto troch prvkov \cite{availability}. Nutno poznamenať, 
že to úplne neplatí, pretože nežiadaným znemožnením prístupu k zdrojom sa ich neporušenosť a zabezpečenosť 
proti neoprávnenému prezeraniu, či úprave, stáva bezpredmetná. O dostupnosť sa teda opierajú všetky ďalšie
bezpečnostné predpoklady, ktoré má systém napĺňať.

\subsection{Kľúčový zabezpečovatelia dostupnosti}
Komunikačné technológie často tvoria chrbtovú kosť väčšiny moderných biznisov pričom ich hlavnou úlohou je 
sprostredkovanie informácií naprieč organizáciou a podieľajú sa na riadení podnikových procesov. Okrem 
ľudského kapitálu sa spolieha na tri kľúčové prvky: \emph{softvér, hardvér a počítačovú sieť} 
\cite{availability}. 

\paragraph{Softvér:}
Softvér je najkritickejším komponentom spomedzi vymenovaných, pretože na základe príkazov v kóde programov 
je ovládaný hardvér a sieťové zariadenia. Všetky potenciálne útoky a ich dopady musia byť riešené primárne 
na úrovni softvéru. Na napadnutie sú využívané zraniteľnosti v ovládacích a regulačných mechanizmoch 
systému, sprístupnené prelomením nedostatočne zabezpečeného verejného rozhrania služieb. Najčastejším cieľom 
útočníka je dosiahnutie kontroly nad zariadením alebo vyvolaním chaosu vo fungovaní prevádzky. 
 
Zlyhanie programového vybavenia nemusí byť iba v dôsledku nepriaznivých vonkajších vplyvov, ale tiež sa 
prejavujú chyby spôsobené nekorektným návrhom alebo implementáciou systému s odchýlkami od
požadovaného správania, majúce nepriaznivý vplyv napríklad na dostupnosť konkrétnej stránky. Tieto chyby sú 
vnesené neúmyselne najčastejšie programátorom. Počas behu aplikácie môžu nastať zlyhania operačného 
prostredia zapríčinené nedostatkom pamäte pri alokácii, zaplneniu diskového úložiska alebo uviaznutím 
systému. 

\paragraph{Hardvér:}
Poruchy hardvéru bývajú zriedkavejšie, ale o to podstatne fatálnejšie pre celkový chod, keď
výpadok nie je adresovaný redundanciou komponentov. Prinavrátenie do funkčného stavu znamená výmenu
zariadenia za iné prevádzkyschopné, či už dočasnou úpravou fyzickej infraštruktúry alebo neodkladnou 
montážou náhrady. Duplikáciou napr. diskov cez RAID dosiahneme síce vyššiu dostupnosť ale za cenu integrity 
z dôvodu zdvojením dát \cite{availability}, preto je vždy potrebné mať na pamäti vyváženosť bezpečnostných 
vlastností navzájom.

\paragraph{Sieť:}
Obmedzením počítačových sietí je ich priepustnosť podmienená šírkou prenosového pásma a réžiou 
spotrebovanou na obsluhu zvolených komunikačných protokolov. Problémy nastávajú v situáciach, kedy vznikne 
zahlteniu na sieťovej linke. Za normálnych okolností rešpektujú uzly v sieti signalizáciu zaznačenú do 
paketov a prispôsobia rýchlosť vysielania, čím sa za istý čas uľaví náporu. Útočník, ktorý chce saturovať 
serverové pripojenie to  pochopiteľne nerešpektuje a preto by sa nadmerná premávka mala presmerovať a 
filtrovať. Pokiaľ bude smerovač vystavený niekoľkonásobnej záťaži než je schopný podporovať určite spôsobí 
straty paketov a zvýšenú latenciu. Okrem spustenia čo najväčšieho toku paketov do siete sa zneužívajú aj
synchronizačné očakávania komunikácie, konkrétne napríklad stavový automat protokolu TCP.

\subsection{Dôvody viktimizácie prevádzkovateľov}
Nedostupnosť služieb máva za následok, buď priame škody v podobe finančných strát alebo vplýva na 
pošramotenie nadobudnutej reputácie u klientov, ktorí si zlyhanie môžu spájať so stratou spoľahlivosti a 
dôveryhodnosti služby. Poškodenie reputácie sa radí so 47\% medzi najväčšie obavy firiem počas 
kybernetického útoku \cite{radware-ddos}. Ďalšími zreteľmi starostí sú potenciálna strata ziskov s 21\%, 
obmedzená dostupnosť s 12\% a zníženie produktivity počas útoku na 7\%. Čím väčší poskytovateľ a 
používanejšia webová stránka, tým rozsiahlejšie sú potenciálne dopady pri neočakávanej vyradení z prevádzky. 
Zároveň dochádza k adekvátnemu navýšeniu zriadenej odolnosti investovanej do predchádzania a aktívnej 
defenzívy proti útokom. 

Motivácie a dôvody stojace za aktivitami úmyselne narúšajúcimi dostupnosť zvolených obetí v podobe
webových aplikácií sa líšia od prípadu k prípadu, ale dajú sa zhrnúť do nasledujúcich kategórií 
\cite{why-attack} \cite{ddos-attacks}:
\begin{itemize}
\itemsep0em 
\item \emph{Kapitálové zisky} - nadobudnutie finančnej odplaty od objednávateľa alebo nekalé snahy o 
potopenie konkurencie v ekonomickej súťaži predstavuje významný hnací faktor pre útočníkov. 
Hlavným zámerom je lepšie peňažné zabezpečenie sa nehľadiac na taktiky vynaložené na tento účel. 
Úspešná realizácia vyžaduje značné technické zručnosti, pretože firmám, ktoré sú hlavných cieľom,
ide o veľa. 

Zvykne sa dotýkať komerčných webstránok alebo služieb finančných inštitúcií (zaznamenané útoky na HSBC, 
BTC a  Ethereum burzy), serverov a sieťových zariadení poskytovateľov webhostingu alebo internetového 
pripojenia (Deutsche Telekom, OVH, Dyn), herných serverov (Steam, Blizzard, EA Sports), emailových serverov 
(Eir). \cite{why-attack}

\item \emph{Pomsta} - prevažne frustrovaný jednotlivci snažiaci sa o odplatu za vnímanú 
nespravodlivosť, ktorú podľa nich prevádzkovateľ pácha. Počas operácie Payback z roku 2010 bol odplatou
hackerskej skupiny Anonymous, za blokovanie stránok s torentami a pirátskym softvérom, útok odoprenia 
služieb na organizácie chrániace autorské práva. V decembri zhodili stránky MasterCard, Visa, Paypal a 
iných, ktoré vydržiavali donácie organizácii Wikileaks, pretože publikovala prísne tajné informácie
americkej vlády.

\item \emph{Ideologické a politické presvedčenie} - útočník sa snaží dať hlasno najavo svoj nesúhlas s 
ideovo protichodnými názormi a postojmi znefunkčnením alebo poškodením platformy prostredníctvom ktorej 
oponent pôsobí alebo šíri svoj svetonázor. Takáto forma útokov sa označuje tiež za hacktivizmus, ktorého 
prívrženci hájia slobodu slova a právo na súkromie proti nadmernému sledovaniu. Pre predstavu napádaných 
webstránok boli prominentné útoky roku 2016 cielené na skupiny ako Black Lives Matter, Ku Klux Klan, 
Wikileaks, oboch prezidentských kandidátov v USA, taliansku a írsku vládu alebo európsku komisiu. 
\cite{why-attack}

\item \emph{Demonštrácia schopností} - ide o experimentálne útoky so zámerom hackera vyskúšať si nové 
techniky alebo predviesť svoje kompetencie.
 
\item\emph{Kybernetický terorizmus} - útočník je súčasťou vojenskej alebo teroristickej operácie s cieľom
poškodenia nepriateľovi. Kritická infraštruktúra štátu prestavuje najčastejšie zasahovaný cieľ.
\end{itemize}

\subsubsection{Teória rutinných aktivít}
Výber primeranej obete má taktiež svoj podiel na úspešnosti škodlivého zásahu útočníka do prevádzky,
pretože ten sa výrazne nelíši od konvenčného zločinu. Medzi obvykle spomínané zdôvodnenia páchania 
kriminality zo sociologickej perspektívy patrí kriminologická teória známa pod názvom 
\emph{teória rutinných aktivít}. Poskytuje predpoklady, aké musí daný subjekt spĺňať nato, aby bol 
zasiahnutý. Vyslovuje, že zločin sa udeje vtedy, keď motivovaný útočník, so sklonmi na páchanie trestnej 
činnosti, príde do stretu s objektom ponechaného bez prítomnosti schopného strážcu \cite{cohen-felson}.

Pokiaľ sú vytvorené priaznivé okolnosti na prelomenie do systému, napríklad predvoleným prepustením 
všetkej sieťovej premávky je veľká šanca, že motivovaný útočník sa bude snažiť o zneužitie a jednoduchšie
je mu umožnené zhodiť tento systém. Na druhej stane pri absencii jediného kritéria, sa buď znižujú šance na
viktimizáciu alebo úplne vylúči. Takáto situácia vie odradiť značné množstvo potenciálnych útočníkov.
Podľa \emph{teórie racionálneho jednania} koná útočník pri zvažovaní realizácie svojho činu racionálne,
hoc sa jedná o obmedzenú rozumnosť a síce z uhľa pohľadu delikventa ide o cieľavedomé rozhodnutie, kde 
pozitívny obnos prevažuje nad možnými rizikami uskutočnenia.

Vhodnosť zamerania sa na zvolenú infraštruktúru pre potenciálneho páchateľa je zachytiteľné kritériami 
\emph{VIVA} (Value, Inetria, Visibility, Accesibility) \cite{why-attack}. V kontexte útokov odmietnutia 
služby sa pod \emph{hodnotou} rozumie dôležitosť rozbitia cieľa pre útočníka, presnejšie či daný internetový 
portál alebo herný server dosahuje dostatočné zisky, aby ich odstavením spôsobil dostatočnú škodu. 
\emph{Zotrvačnosťou} sa myslí odpor, ktorý kladie infraštruktúra voči útoku rozličnými bezpečnostnými 
mechanizmami. Medzi priamočiare praktiky patrí udržiavanie aktualizovanému systému so zaplátanými 
zraniteľnosťami alebo obmedzením počtu dopytov z jednej adresy. Služby s veľkou zotrvačnosťou sú schopné 
ustáť väčší nápor v prípade napadnutia. \emph{Viditeľnosť} predstavuje rozsah v akom je webstránka verejne 
dostupná a známa širšiemu publiku. \emph{Prístupnosť} značí jednoduchosť v dosiahnutí vytýčených sieťových 
uzlov, ktoré majú predstavovať obeť, použitou taktikou útoku bez povšimnutia. Rovnako sa spája so 
schopnosťou nezanechať stopy na mieste činu následkom neprítomnosti mechanizmu monitorovania a detekcie 
narušenia. 

Relatívne vysokou hodnotou, viditeľnosťou a prístupnosťou a nízkou zotrvačnosťou sa cieľ stáva 
exponovanejší a tým žiadanejší pre útočníka na odstavenie. Oproti klasickému zločinu, kedy býva nutná 
prítomnosť páchateľa a obete na jednom mieste, kyberzločin dovoľuje útočníkovi pôsobiť cez internet takmer 
od hocikadiaľ a maskovať sa proti odhaleniu.

\subsubsection{Psychologická predeterminácia útočníka}
Pokiaľ by nejestvovali indivíduá so zámerom druhému spôsobiť ujmu vyplývajúcu z nastolených motivačných
faktorov nebolo by ani potrebné sa výrazne zaoberať zvyšovaním odolnosti informačných služieb, či brániť
voči kriminalite ako takej. Určité vzorce ľudského správania naznačujú, že je prakticky nemožné sa pred 
týmito spoločenskými javmi vyhraniť.

Teória diferenciálnej asociácie tvrdí, že v spoločnosti existujú paralelne tak prosociálne, ako aj asociálne 
normy, postoje a spôsoby správania \cite{heretik}, čím sa vysvetľujú trestné činy dostatočne zabezpečených 
jedincov strednej vrstvy. Dochádza u nich k stotožňovaniu sa s antisociálnymi prístupmi na ceste za osobným 
úspechom. Zároveň páchatelia podvedome zľahčujú následky  sociálneho zlyhania v spojitosti s stanovenými 
antikriminálnymi normami. V snahu neutralizovať svoje konanie popierajú zodpovednosti skrývaním sa za 
bezvýchodiskovosť situácie, neuznávajú význam obete  prenášajúc naň vinu a ohraďujú sa konaním v záujme 
vyššieho princípu. Na základe teórie etiketovania je delikvencia len momentálny stav osobnosti, ktorý 
pramálo súvisí s psychickými vlastnosťami a správaním.

Predpokladom na spáchanie kybernetického zločinu sústrediac sa na snahu o eliminovanie dostupnosti
sú okrem úvodných pohnútok aj isté technické zručnosti útočníka. Keď chce byť pracovník slobodný od 
nadriadených a má, hoc aj nedopatrením, záznam v registri trestov, je niekedy motivovaný vlastnou 
otáznejšou zamestnateľnosťou sa uchýliť k predávaniu alebo prenajímaniu strojov, ktoré sa mu podarí 
kompromitovať, pre niekoho kto má záujem učiniť veľkoplošný útok a vedú ho k tomu zrejme dôvody uvedenej 
skorej. Hacker si časom  vybuduje ilegálny biznis, ktorým si dokáže zarobiť uspokojivý obnos 
\cite{infiltrating-botnet}. Využívané programy sú vymieňané alebo predávané sprostredkovane cez fóra,
sú reklamované a tamojšími administrátormi overované podobne ako bežný komerčný softvér. Na fórach 
prebiehajú diskusie a objavujú sa návody k schodným spôsobom ako si dostupný malvér sprevádzkovať 
a upraviť podľa potreby.

Vykonanie útoku so zámerom obmedziť dostupnosť cudzieho systém je podľa platnej legislatívnej úpravy 
na Slovensku trestným činom podľa \emph{§247a Trestného zákona} a obdoby sú zavedené takisto v iných 
právnych poriadkoch (napr. v USA - Computer Fraud and Abuse Act a 18 U.S.C. § 1030, v Nemecku - 
Strafgesetzbuch: §303b  Computersabotage). Činnosť neoprávneného zásahu do počítačového systému spadá v 
podstate do rovnakej oblasti ako poškodenie cudzieho majetku s trestami odňatia slobody pri preukázaní, od 
šiestich mesiacov až po 10 rokov podľa závažnosti \cite{trestny-zakon}. 

\section{Anatómia útokov Denial of Service}
Internet predstavuje prostredie sprístupňujúce na jednej strane ohromnú kvantitu služieb, ale zároveň 
sprostredkúva útočníkom širokú paletu nástrojov umožňujúcich ich odstavenie. Útoky odoprenia služby
- \emph{Denial of Service (DoS)} - spôsobujú nežiaduci zásah do schopnosti legitímneho 
používateľa na prístup k zdrojom dostupných v počítačovej sieti. Zneprístupnenie sa realizuje vyčerpaním šírky pásma linky alebo systémových prostriedkov obete, či už CPU, operačnej pamäti alebo priepustnosti 
vstupno-výstupných operácií \cite{ddos-attacks}. Pokiaľ sa na útoku podieľa značný počet zariadení označuje 
sa ako distribuovaný DoS skrátene DDoS. 

Útoky typu DDoS sú na internete obrovským problémom napriek snahe výskumníkov vyvíjať neustále lepšie 
metódy obrany v reakcii na stále sofistikovanejšie modifikácie techník útokov. V architektúre internetu 
prevládajú prvky so zameraním skôr na efektivitu a spoľahlivosť prenosu paketov medzi koncovými uzlami, než 
na silné zabezpečenie detailnou kontrolou prenášaného toku. Z distribuovanej povahy a autonómie 
administrácie nezávislých samostatných sietí, z ktorých je Internet zložený, a ktoré si riadi 
do veľkej miery každý poskytovateľ pripojenia zvlášť, by bola na systematické celoplošné politiky nevyhnutná 
ťažko dosiahnuteľná širšia dohoda. Zároveň platí, že akokoľvek je cieľový systém ochránený stále závisí od 
úrovne zabezpečenia ostatných uzlov v sieti. Okrem rôznorodého vynútenia pravidiel je ďalším predpokladom na 
degradáciu služieb obmedzené výpočtové zdroje každej entity na trase, hlavne ide o limitované kapacity 
vyrovnávacích pamätí.

Myšlienka uskutočnenia DoS útoku je pomerne priamočiara. Pokiaľ útočník disponuje väčšou celkovou rýchlosťou 
pripojenia, je schopný preťažiť linku obete a tým spomaliť spracovanie oprávnených požiadaviek. Prenesene sa 
uplatňuje zásada, že silnejší pri súboji vyhráva. Strana disponujúca lepším pripojením spravidla predurčí 
stav dostupnosti služby v kritických momentoch. Spustenie útoku iba z jedného akokoľvek výkonného stroja
je pre útočníkov nevýhodné, pretože zmarenie zlovoľnej činnosti spočíva jednoducho vo vyčítaní adresy 
pôvodcu odchytávaním premávky, jeho následne zablokovanie a pridanie na čierne zoznamy. 

\subsection{Botnet}
Centralizované prevedenie útoku odoprenia služby je z dnešného pohľadu nepriechodné. Najčastejšie sa preto 
na znefunkčnenie služby uplatňuje taktika ovládnutia rozsiahlej skupiny zraniteľných počítačov, ktoré dokáže 
útočník ovládať na diaľku a nasadiť do želanej ofenzívy. Napadnuté zariadenia ani ich užívatelia 
častokrát netušia, že sa stali súčasťou takéhoto zoskupenia, ktoré sa označuje ako \emph{botnet}. 
Počítač slúžiaci útočníkovi na naplnenie nekalých úmyslov vzdialeným vykonaných povelov je tzv. \emph{bot}, 
\emph{zombie}, či \emph{dron}. Boti sa správajú ako hybrid viacerých kybernetických hrozieb s pridanou 
hodnotou komunikačného  kanála so schopnosťou koordinácie cez ovládacie miesta. Šíria sa podobne červom, 
skrývajú sa pred detekciou ako vírusy a obsahujú útočne metódy toolkitov \cite{zombie-roundup}. Vlastník 
armády botov je tzv. \emph{botmaster} alebo \emph{pastier - herder}. Neprávom nadobudnuté výpočtové 
prostriedky riadi prostredníctvom command and control (C\&C) infraštruktúry.

\subsubsection{Komunikácia s botmi}
DDoS útočné siete používajú spravidla tri typy architektúry: \emph{Agent-Handler}, \emph{Internet Relay Chat 
(IRC)} a \emph{webovú architektúru} \cite{ddos-attacks} \cite{botnets}. Model Agent-Handler pozostáva z 
klientov - útočníkov, ktorí sa pripájajú na tzv. handler so zaneseným softvérovým vybavení na zisťovanie 
stavu a koordináciu agentov. Umiestňuje sa spravidla do zariadení s veľkým objemom sieťovej premávky a ich 
strategický výber umožňuje výrazne kamuflovať podozrivú komunikáciu. Terminológia \emph{handler} a 
\emph{agent} sa zvykne zamieňať s \emph{master} a \emph{démon}. 

Medzičlánkom preposielania povelov od botmasterov sa rovnako môže stať verejný IRC server. Vtedy sa jedná o 
architektúru založenú na protokole Internet Relay Chat. Pôvodný účel využitia botov spočíval pri
asistencii moderovania rušných četových miestností IRC kanálov \cite{zombie-roundup}. Jedným z prvých bol 
bot Eggdrop napísaný už v roku 1993. V tom čase začali vznikať boti so zámerom útočiť na ostatných 
používateľov a IRC servery. Dovoľovali útočníkovi ukrytie sa za aktivity bota alebo dokonca za botov na 
viacerých počítačoch, ktorý neboli k útočníkom priamo vystopovateľný. Tým bolo umožnené napádať čím ďalej 
väčšie ciele. 

Agenti sa po pridaní k botnetu ohlásia na dezignovaný IRC kanál a ďalej prijímajú a posielajú správy 
cezeň. Tento spôsob je lákavý pre jednoduchosť komunikácie v podobe krátkych textových správ príkazov a 
dostatočnú anonymitu bez silnej autentifikácie. Útočník nemusí udržiavať zoznam dostupných agentov, keďže po 
prihlásení sa na server vie zobraziť všetkých podriadených botov. Pre zložitejšie odhalenie napomáha 
využitie známych portov pre IRC (6667/TCP), pomerne veľká prevádzka na známych IRC serveroch a technika 
\enquote{preskakovania medzi kanálmi} (channel hoping), kedy botmaster využíva zvolený IRC kanál iba na 
krátke obdobie.

Najpoužívanejším modelom je síce pre svoju flexibilitu IRC forma komunikácie, ale v posledných rokoch sa 
objavujú botnety založené na webových aplikáciach. Boty posielajú webovému serveru pravidelne informácie o 
svojom stave. Ovládané sú cez komplexné PHP skripty a komunikácia s agentami dokáže byť šifrovaná cez TLS a 
skrývať sa za bežnú webovú prevádzku na portoch 80/TCP, 443/TCP a tým odolávať tým bežným sieťovým filtrom. 
Narozdiel od IRC spočíva ich nesporná výhoda v nemožnosti únosu botnetu od svojho pôvodného tvorcu únosom 
četovej miestnosti.

\subsubsection{Šírenie replikáciou škodlivého kódu}
Nehľadiac na výber spôsobu komunikácie agentov so svojim command and control uzlom, musí ich sieť byť 
dostatočne rozsiahla na to, aby spôsobila znateľnejší dopadu na webové služby. Zároveň by mala disponovať 
metódami vlastnej replikácie sa na priľahlé napadnuteľné počítače. Priebeh rozširovania vplyvu botnetu
nad zväčšujúcou sa skupinou hostiteľov sa odohráva v postupných fázach. 

V prvom rade musí dôjsť k objaveniu zraniteľných hostov, potenciálnych budúcich botov. Útočník si
môže vytipovať vhodnú známu obeť a pokúsiť sa o prevzatie kontroly manuálne, systematickým skúšaním
prelomenia známych zraniteľností konkrétneho systému. Automatizované skripty, ktoré sú umiestňované do už 
nakazených počítačov sa nepotrebujú vopred špecificky zacieliť, ale dokážu si poskladať zoznam IP adries, 
ktoré bude postupne navštevovať a preverovať preddefinované nezaplátané bezpečnostné diery. 

Skenovanie môže prebiehať \textbf{náhodne} \cite{ddos-anatomy-2004}, kedy každý kompromitovaný uzol v sieti 
generuje postupnosť ľubovoľných IP adries. Technika je použiteľná iba pri IPv4, pretože pri hustote 
rozloženia obsadených IPv6 adries by bol tento postup výrazne neefektívny. Náhodné skúšanie hostov
vytvára veľký objem podozrivej sieťovej premávky smerovanej akiste medzi vzdialenými sieťami, 
ktoré normálne nekomunikujú, čím sa zvyšuje šanca na odhalenie takejto aktivity. Keďže nedochádza 
pri skenovaní k synchronizácii medzi infikovanými počítačmi rastie množstvo duplicitných dopytov
na rovnaký už preverený koncový uzol s ich zväčšujúcim sa počtom.

Obdržaním zoznamu počítačov (\textbf{hitlist}) s ľahko prelomiteľnou obranou dokáže botnet usmerniť svoje 
šírenie. Známym vyhľadávačom verejných adries IoT zariadení s konektivitou k Internetu a prehľadom
známych bezpečnostných dier je \verb|shodan.io|. Kolíziam sondovania sa zabraňuje prerozdelením celého 
hitlistu na menšie časti, čím sa zabezpečí, že každý agent overí stroje z presne určeného rozsahu. Nevýhoda 
spočíva v nutnom zostavení celého zoznamu predtým než dôjde k samotnému rozširovaniu útočnej siete. Dôležité 
je zvolenie vhodnej veľkosti jeho dielov na preposielanie. Ak je zoznam rozsiahly tvorí sa značná sieťová 
premávka, krátky zoznam zapríčiní malú finálnu populáciu agentov. 

\textbf{Topologické skenovanie} nasleduje prirodzene vznikajúce komunikácie objavujúce v sieti, aby
sa dosiahlo presnejšie splynutie s bežným tokom paketov. Nakazený hostiteľ v podobe webového servera 
pošle do prehliadača klientov škodlivý kód a za správnych okolností sa ten dokáže dostať na iné webové
servere, ktoré klient prezerá. Spoliehaním sa na správanie používateľov sa výrazne znižuje rýchlosť a 
úplnosť ovládnutia vyhovujúcich obetí a útočník nedokáže šírenie počítačového červa regulovať.

Predošlé varianty skenovania je užitočné upraviť na prehľadávanie cieľov \textbf{v lokálnej podsieti}, 
čím sa dajú nakaziť náchylné počítače za firewallom a agent pritom neprezrádza svoju lokáciu využívaním
nadmernej intersieťovej výmeny správ.   

Nachádzanie vektorov prieniku počas prechádzania zoznamom adries je uskutočňované, buď horizontálne,
napríklad preverovaním rovnakého otvoreného portu, či mierenej zraniteľnosti naprieč všetkými cieľmi, alebo
sa koná vertikálne a síce testovaním širokého spektra malvérom pribalených utilít snažiac sa vniknúť
dnu hocako. S vykonávaním vybranej metódy pomalým tempom má útočník príležitosť zostať nebadaný po dlhšiu
dobu a ponúka sa mu čas na preverenie možností získania kontroly nad systémom. Po fázach náboru a 
vykoristení nového bota sa naň prenáša škodlivý kód pochádzajúci z centrálneho úložiska (červ 1i0n) alebo
sa siahne zo zariadenia, ktoré bol pôvodcom nákazy v predošlom kroku, tzv. \enquote{back-chaining} (červy
Morris, Ramen) \cite{ddos-anatomy-2004}. 

\subsection{Klasifikácia typov DDoS útokov}
Útok odoprenia služby závisí od schopnosti čo najväčšej alebo špeciálne zameranej sieťovej 
premávky, aby informačná služba neakceptovala požiadavky legitímnych žiadateľov. Odohráva sa
privlastnenie si celej vyhradenej linky alebo výpočtového výkonu prevádzkovateľa útočníkom. 
Ak je vyústením prevalcovanie serverovej infraštruktúry obete nedokonalosťou zabezpečovacích 
mechanizmov dochádza k zrušeniu dostupnosti s následkami už uvedenými. Aby sme porozumeli metódam 
efektívnej obrany je nevyhnutné zatriediť a kategorizovať objavujúce sa
hrozby, s ktorými sa ciele DDoS útoku vedia stretnúť. V literatúre existujú rozličné taxonómie 
separujúce problematiku z rôznych uhlov pohľadu \cite{ddos-attacks} \cite{botnets} \cite{ddos-anatomy-2004} 
\cite{csirt-ddos}.

Základné rozdelenie DDoS útokov spočíva v identifikácii ich primárneho vektora. Webová aplikácia
býva zneprístupnená, buď vyčerpaním šírky prenosového pásma hrubou silou záplavy paketov, alebo
vyplytvaním systémových prostriedkov sémantickým útokom na komunikačný protokol.

\textbf{Volumetrické útoky} (veľkoobjemové útoky) saturujú kapacitu linky rozmanitou plejádou nálože. 
Spoločným menovateľom je technika flooding (záplava). Populárnou formou útoku je posielanie UDP datagramov 
na náhodné porty s úmyslom zapríčiniť overovanie, či sú porty otvorené a spôsobiť reakciu 
servera signalizačnými správami ICMP Destination port unreachable. So snahou donútiť systém, aby sa venoval 
predovšetkým záškodníckym spávam útočníka, pracuje tiež záplava paketmi ICMP Echo Request (Ping) 
s následnou odpoveďou ICMP Echo Reply. Na VoIP služby je účinným SIP Flood, ktorý zaplaví SIP proxy s
falošnými správami pre začatie hovoru SIP INVITE \cite{botnets}. Webový aplikačný server je možné zahltiť 
záplavou HTTP(S) požiadaviek GET alebo POST na náhodné alebo existujúce URI webstránky. Dopytovanie sa na 
neexistujúcu cestu okamžite vráti stavový kód rádu 400 pre  chybu klienta, ale rovnako sa bude server 
musieť zaoberať spracovaním takejto požiadavky, len sa stáva jednoduchšie pozorovateľnou z prístupových 
logov. 

\textbf{Protokolové útoky} sa priživujú na zraniteľnosti v návrhu komunikačného protokolu na transportnej
až aplikačnej vrstve OSI, ktoré spoliehajú na priebežné ukladanie stavových informácii o naviazaných
reláciach. Rozšírené sú taktiky na zneužitie časovače stavového automatu protokolu TCP a príznakov v TCP 
segmentoch, ktorými sa odosielateľ a prijímateľ dohadujú na priebehu výmeny správ. 

Počas TCP SYN Flood je doručené také množstvo podnetov na otvorenie spojenia segmentami s príznakom SYN, 
ktoré vyústi v zaplnenie pamäte vyhradenie na uchovávanie aktívnych relácii. Server je povinný pri 
zahajovaní TCP spojenia cez 3-way handshake a obdržaní SYN odoslať SYN+ACK a počkať stanovenú dobu. Timeout 
býva dostatočný na to, aby dokázal útočník ponechať tabuľku relácií zaplnenú iba svojimi podvratnými 
požiadavkami. Schodnou ochranou je zavedenie tzv. TCP Cookie. Systém po prijatí TCP SYN odošle TCP SYN+ACK a 
nevytvorí v pamäti žiadnu reláciu \cite{csirt-ddos}. Po prijatí právoplatnej odpovede TCP ACK sa spätne dopočíta TCP sekvencia paketov a až vtedy sa zaháji spojenie. 

TCP RST útok sa zameriava na rušenie nadviazaných spojení medzi serverom a klientmi, kedy však je
nutné poznať zdrojovú IP adresu klienta, pretože útočník háda začiatkom konverzácie náhodne započaté 
sekvenčné čísla. Ak uspeje preberie reláciu a zruší ju. Za bežných okolností je také niečo ťažko 
spáchateľné, lebo TCP spojenia zvyknú mať krátke trvanie a vznikajú ad-hoc.

Nastavením príznaku PSH je serveru nanútené okamžité vyprázdnenie vyrovnávanie pamäte klientovi
a odoslanie potvrdzujúcej správy ACK. Pri enormnej hromade takýchto výziev nebude schopný server
vybavovať ďalšie požiadavky, čím dôjde k zrušeniu dostupnosti webových a podobných služieb poskytovaných
z daného bodu.

Zlomyseľným zásahom do riadenia toku TCP spojenia predchádzajúceho zahlteniam, presnejšie vyžiadaním a 
udržiavaním nulovej veľkosti okna príjemcu, sa vie útočník obsadiť všetky dostupné spojenia v tabuľke
spojení a tým znemožniť nadviazanie komunikácie so serverom ostatným. Určenie veľmi malej nenulovej
veľkosti okna spôsobí rozdrobenie odpovedí na veľmi malé fragmenty. Prevenciou býva zapojenie Nagelovho
algoritmu (RFC 896) do TCP implementácie, ktorého úlohou je zamedziť veľkej réžii pri posielaní 
miniatúrneho payloadu.

Na relačnej vrstve modelu OSI je vďačným protokolom na útoky spotrebujúce značný výpočtový výkon Secure
Sockets Layer (SSL/TLS). Keďže majorita webových aplikácií v súčastnosti používa HTTPS
je dôležité si uvedomiť, že proces šifrovania spolu s réžiou pri výmene kľúčov v SSL handshaku predstavuje 
pre server násobnú náročnosť oproti klientovi. Znovu sa naskýta prostý útok záplavou iniciácií TLS spojenia
alebo opätovné dohodnutie SSL komunikácie (renegotiation), ktorá zvykne zahŕňať zmenu parametrov šifrovania
alebo vyžiadanie certifikátu servera. Riešením je blokovanie takýchto požiadaviek alebo \enquote{SSL 
offloading} do špecializovaného hardvéru \cite{csirt-ddos}.

Spotrebovanie všetkých ponúkaných spojení HTTP protokolu aplikáciou webového servera sú preferované
tzv. \enquote{low and slow} útokmi. Agent sa maskuje akoby za veľmi pomalú rýchlosť pripojenia,
no v skutočnosti zámerne rozdrobuje svoj dopyt na krátke fragmenty a posiela ich s významným oneskorením,
aby držal spojenie otvorené čo najdlhšie. Dôvody existujúcich obmedzení tkvejú v maximálnom počte 
súborových deskriptorov procesu alebo únosnej hladine bežiacich procesov. Zástupcom tejto skupiny
útokov je Slowloris a R-U-Dead-Yet? (RUDY). Z dôvodu malej generovanej premávky, prechádza pomerne ľahko 
bez povšimnutia, pretože nemá dopad na iné atribúty systému.

Rovnako ako botnety slúžia na zitenzívnenie devastačného prúdu paketov, tak môžu nepriamo zapájať 
do útoku aj malvérom nenakazené počítače technikami odrazu a zosilnenia. Útoky s odrazom (RDoS a DRDoS) 
zapríčiňujú poslanie paketov s podvrhnutou zdrojovou adresou cieľa útoku záchytným bodom (pivotom).
V domnienke správnosti pôvodcu správy sa odpoveď doručí v konečnom dôsledku na obeť. Samo o sebe to nemá až 
taký význam, okrem odklonenia nevyhnutného prúdu odpovedí od skutočných spúšťačov požiadaviek na tretie
strany. Nastavením cieľovej IP adresy na broadcastovú adresu lokálnej podsiete (L2 alebo L3 OSI) sa
útok zosilní, pričom zasiahne všetky počítače v spoločnom broadcastovom segmente siete. Tiež je priechodné 
odrazenie útoku od viacerých reflektorov. Na týchto princípoch fungujú útoky Smurf a Fraggle.

Domain Name System (DNS) amplifikačné útoky využívajú podstatu odrazeného útoku, ale obsahujú obohatenie
zaručeného nárastu veľkosti DNS odpovede voči dopytu. Faktor zväčšenia DNS query response sa pohybuje
od 1,1 pri jednom A zázname (example.net), 2,75 v prípade troch AAAA záznamoch (youtube.com) alebo 
dokonca 3,5 (yahoo.com) pri siedmich štvor-áčkových záznamoch. Protokol DNSSEC ponúka
cez otvorené rekurzívne resolvery až 30-násobnú amplifikáciu \cite{csirt-ddos}, 
z dôvodu početnosti vrátených NS záznamov a digitálnych podpisov v DS a RRSIG záznamoch. 
Uvedenými príkazmi sme objavili odpoveď s 24-násobným zväčšením a podobne iné domény dosahovali
amplifikácie bežne v rozsahu 13 - 18-krát:
\begin{lstlisting}
dig +dnssec +trace opendns.com
dig +dnssec @b.root-servers.net opendns.com
\end{lstlisting}

Každoročný prehľad v trendoch kybernetických hrozieb publikovaných európskou inštitúciou ENISA
konštatuje, že takmer 80\% všetkých DDoS útokov v treťom kvartáli 2019 boli TCP SYN záplavy 
\cite{enisa-ddos}, stávajúc sa najpopulárnejším typom útoku spolu s DNS odrazenou amplifikáciou. 
V apríli 2019 bol zaznamenaný SYN Flood útok s prietokom až 580 miliónov paketov za sekundu. Vyskytujú
sa hlavne multivektorové útoky, čím je ich zdolanie komplexnejšie. Zároveň dominovali útoky kratšie
ako 10 minút, ktorých bolo 84\% zo zaznamenaných. Celkovo došlo k nárastu v počte nahlásených útokov
o 241\% oproti rovnakom období predošlého roku. \cite{enisa-ddos}.

\subsection{Ochrana spevnením sieťovej ochrany}
Obranné mechanizmy na úspešné zvládnutie útokov odoprenia služby rozlišujeme primárne podľa
úrovne pripravenosti reakcie na \textbf{proaktívne} a \textbf{reaktívne} stratégie
\cite{ddos-anatomy-2004}.

\textbf{Prevenciou} sa zabezpečuje systém proti prieniku priebežným monitorovaním
a pravidelným sťahovaním a inštaláciou bezpečnostných záplat. Súčasne sa budujú
bezpečnostné politiky organizácie, ktoré rátajú s klasickými vektormi DDoS útokov a premýšľa sa 
nad adekvátnym minimalizovaním dopadu hrozieb s nimi spojenými. Prakticky osvedčenými 
riešeniami sú \enquote{resource accounting}, čiže účtovanie a limitovanie počtu vyhradených spojení 
pre každú IP adresu pristupujúcu k službe pri maximálnej frekvencii odpovedí od servera, alebo sa využíva 
\enquote{resource muliplication}, kedy sú zdroje systému duplikované na viaceré zariadenia a prichádzajúca 
záťaže je vyvažovaná medzi nimi.

\textbf{Reaktívne spôsoby} sa usilujú o zmiernenie útoku počas jeho konania. Pre automatizované
riadenie defenzívy by mali byť obranné prvky schopné detekcie útoku odlíšením od typickej sieťovej 
premávky s ohľadom na odstránenie vlastnej chybovosti pri identifikácii falošných poplachov
a prehliadnutí škodlivých činností. Zisťovanie prítomnosti pokusov na odoprenie služieb
sa sleduje rozpoznávaním podozrivých vzorov podľa analýzy predošlých útokov alebo
pozorovanie anomálií v komunikácií. Strážny komponent je sústredený na preddefinované scenáre alebo
sa natrénuje na bežnej premávke a spustí varovanie po prekročení prahových hodnôt.

Pri oboch stratégiách je ideálne aplikovať taktiku viacerých línií obrany, ktoré sa osvedčili už 
za dávnych  čias pri fyzickej obrane pevností. Bezpečné prostredie má ľudí a aktíva chrániť pred ujmou a 
stratou. Staroveké civilizácie si na ten účel stavali obranné pevnosti, ktoré mali poskytnúť ochranu pre 
veci vnútri nich. Už Mykénska civilizácia uplatňovala štyri zóny fyzickej obrany pred nepriateľom 
\cite{physical-security}. 

Prvou líniou obrany bola priekopa alebo rieka s premosteniami, za ktorých paralelu je možné vo virtuálnom 
svete považovať vzdialene spustiteľné čierne diery (RTBH) reaktívne chrániace autonómny systém (AS) 
regionálne (AS) alebo vcelku. Za priekopou sa v druhej línii nachádzal obvodový múr a vstupné brány, kde je 
sieťových ekvivalentom snáď Ingress, Egress a ACL filtrovanie premávky na smerovačoch alebo 
zapojením systémov detekcie a prevencie narušenia - IDS a IPS. Vnútri pevností obkolesovala menšia vnútorná 
stena svätostánky a palác, čo sa dá pripodobniť k firewall pravidlám operačného systému, na ktorom služba 
beží. Posledná najvnútornejšia obranná stena stála pred pokladom v kráľovskej štvrti. Táto úroveň sa dá 
stotožniť s bezpečnostnými mechanizmami webového servera v podobe Web Application Firewall (WAF) a podobných
modulov ako existujú pre Apache s ModSecurity a ModEvasive.

Pozorujeme, že smerom ďalej do centra chráneného webového  servera, tým dokážeme odolávať efektívnejšie 
rozsiahlejším útokom, keďže spravidla chrbticová sieť disponuje najsilnejšími prostriedkami, ale na druhej 
strane sa jej problémy pôsobiace nižšie po prúde až v takej miere nedotýkajú.

\subsubsection{Remotely Triggered Black Hole} 
Nežiadúcu premávku je teda vhodné zlikvidovať na okraji 
autonómneho systému vzdialene spustiteľnou čiernou dierou (RTBH). Nahlásením zdrojov alebo 
častejšie cieľa útoku vie správca siete vložiť do spúšťacieho smerovača statickú cestu na virtuálne 
rozhranie \emph{Null}, ktoré spôsobí zahodenie paketov (RFC 5635). Postihnutá služba bude síce odrezaná 
od Internetu,  čiže útočník dosiahne zamedzenie dostupnosti pre ostatných používateľov, ale zmiernia sa 
negatívne dopady na zvyšnú infraštruktúru. Na edge routeroch sa musí vopred nakonfigurovať statická cesta 
pre ďalší skok stanovený za black hole a zakázať preň odpovedanie o nedosiahnutí cieľa cez ICMP:
\begin{lstlisting}
ip route 192.0.2.1 255.255.255.255 Null0
interface Null0 
    no ip unreachables
\end{lstlisting}
Na trigger smerovači sa musí aktivovať BGP politika na presmerovanie cesty k obeti 
po obdržaní vloženej statickej cesty so spúšťacou značkou (tagom) a povolenie jej distribúcie
medzi iBGP peerov a zároveň nepropagovanie mimo autonómneho systému \cite{cisco-rtbh}:
\newpage
\begin{lstlisting}
route-map blackhole
    match tag 66
    set ip next-hop 192.0.2.1
    set origin igp
    set community no-export
router bgp 65535
    redistribute static route-map blackhole
\end{lstlisting}
Poskytovateľ pripojenia má potom možnosť iniciovať zablokovanie premávky na obeť známej IP adresy
cez pravidlo na spúšťacom smerovači, ktoré všade presmeruje, v tomto príklade, adresu \verb|172.5.23.1| 
značkou 66 na ďalší skok \verb|192.0.2.1| predstavujúci čiernu dieru. Uvedená statická cesta musí byť 
po skončení odobratá zo spúšťacieho routera, ktorý rozpošle BGP route withdrawal svojim iBGP peerom:
\begin{lstlisting}
ip route 172.5.23.1 255.255.255.255 Null0 tag 66
\end{lstlisting}

\subsubsection{IP Spoofing}
Na zabránenie spätnej väzby paketov vracajúcej sa naspäť botom útočníka, zvykne pri záplavových útokoch 
vložiť do paketu sfalšovanú zdrojová adresa. Vychádza z potreby útočníka zostať v anonymite a prípadne 
zmiasť bezpečnostnú obranu vzbudením dojmu, že nadmerná premávka pochádza od rozptýlených podnecovateľov. 
Spoofing sa hodí u reflektorových útokov alebo v situáciach, kedy si útočník praje, aby bol vinený za 
iniciátora určitý počítač. Prostredníctvom zapojenia Ingress a Egress filtrovania sú poskytovatelia 
pripojenia na trase schopný zabrzdiť škodlivú premávky takmer v zárodku. Ukradnuté zdrojové adresy sa 
pre navodenie dôveryhodnosti vyberajú z rozsahu verejne smerovateľných IP adries (RFC 1918, RFC 3330). 
Privátne adresy sa totiž  štandardne zahadzujú na smerovačoch registrovaním príslušných ACL pravidiel ako 
realizácia \textbf{Egress filtrovania} Skrátená ukážka odhodí pakety z jedného súkromného rozsahov a 
prepustí ostatné \cite{cisco-spoofing-rules}: 
\begin{lstlisting}
access-list 110 deny ip 192.168.0.0 0.0.255.255 any
access-list 110 permit ip any any
\end{lstlisting}

Podvrhnuté IP adresy sa rozlišujú podľa techniky ich selekcie. Najmenej sofistikovaný postup spočíva vo 
vygenerovaní náhodného 32-bitového čísla, ktoré bude predstavovať spoofovanú zdrojovú adresu. 
Vymyslená IP adresa nesmie byť úplne svojvoľná, ale pochádzať z platných podsieti, pre ktoré router má 
záznam v smerovacej tabuľke alebo sa môže nachádzať niekde po ceste k obeti. \textbf{Ingress filtrovanie}
(RFC 2827) totiž povoľuje v striktnom režime smerovať iba pakety, pre ktoré existuje vo FIB (Forwarding 
information base) tabuľke mapovanie reverznej cesty (Reverse Path Forwarding) cez rovnaké rozhranie. Ak 
nedokáže byť zabezpečené symetrické  smerovanie potom \enquote{loose} mód dovoľuje akceptovať spätnú cestu 
cez ľubovoľné rozhranie smerovača:
\begin{lstlisting}
ip verify unicast reverse-path list           # Strict mode
ip verify unicast source reachable-via any    # Loose mode
\end{lstlisting}
Naproti bežnej predstavy prevláda snaha o zúžitkovanie validných zdrojových adries botov, tam kde 
je to uskutočniteľné \cite{ddos-anatomy-2004}.

\subsubsection{Rate limiting a Firewall na Linuxe}
Zníženie záťaže na systémové zdroje pridelené na tvorbu paketov odpovede je dosiahnuteľné cez 
\emph{rate limiting}. Linux oplýva vrámci \emph{procfs} (\verb|/proc/sys/net/ipv4/|) premennými na 
obmedzenie rýchlosti pri odpovedaní maskou zvolenými ICMP správami \verb|icmp_ratelimit|, 
\verb|icmp_ratemask| a \verb|icmp_echo_ignore_broadcasts|\footnote{\url{https://
man7.org/linux/man-pages/man7/icmp.7.html}}. Takisto sa odporúča ponechať pravidlá firewallu na najmenšej
úrovni priepustnosti smerom dnu za každých okolností ako whitelist.
Ak napríklad webová aplikácia beží na porte HTTPS/443 a administrácia vstupuje na server cez SSH/22 zo 
stáleho rozsahu adries intranetu, môže jednoduchá konfigurácia firewallu vyzerá nasledovne 
\cite{csirt-hardening}:
\begin{lstlisting}
# Pravidlá pre povolenie prichádzajúcej komunikácie cez SSH, HTTPS
iptables --append INPUT --protocol tcp --dport 443 --jump ACCEPT
iptables --append INPUT --protocol tcp --dport 22 \ 
         --source 192.168.0.0/24 --jump ACCEPT
# Politiky reťazí pravidiel sú zahodiť všetko okrem odchádzajúcich paketov
iptables --policy INPUT DROP
iptables --policy FORWARD DROP
iptables --policy OUTPUT ACCEPT
\end{lstlisting}

\section{Škálovanie webových aplikácií}
Rozvrhnutie architektúry na nasadenie webovej aplikácie, ktorá ustojí legitímnu premávku, ale tiež 
škodlivo nadmerný prúd vykonštruovaných požiadaviek, vyžaduje balans medzi sústredením prostriedkov na 
jednom mieste do homogénneho monolitu a ich parcelovaním znásobením počtu inštancií. Návrhár musí 
nájsť kompromis medzi obstaraním silnejšieho počítača alebo viacerých počítačov. Priklonenie 
sa k vybranej alternatíve predurčí budúce komplikácie pre rozšírenie platformy hosťujúcej webovú 
službu. 

\textbf{Vertikálne škálovanie} spočíva v navyšovaní výpočtovej sily stroja pridaním, urýchlením 
alebo zväčšením kapacity hardvérových komponentov - procesora, pamätí, diskového poľa, či sieťovej karty - 
utesňujúc vzájomnú väzbu dielcov a očakávajúc ich nevyhnutnú vzájomnú kompatibilitu. Uľahčuje sa tým
údržba a kontrola systému, zachovanie konzistentnosti dát bez nutnosti navyše častí na
zabezpečenie ich integrity. Zároveň sa to odráža na nižšej energetickej spotrebe voči 
mnohým spoločne rovnako výkonným duplikátom. Všetko za cenu o poznanie vyšších obstarávacích nákladov tohto 
kompaktného balíka s výhľadom na obmedzení rozsah pre upgrade, vytvárajúc jediný bod zlyhania pre prípadný 
celkový výpadok poskytovanej aplikácie.

\textbf{Horizontálne škálovanie} je naproti vertikálnemu oveľa odolnejšie proti zlyhaniu, keďže sa spolieha 
na viacero paralelne bežiacich zariadení. Komponenty sú lacnejšie a jednoduchšie na upgrade, 
pretože pádom systému servera nepostihneme dostupnosť služby ako takej. Nevyhnutnosťou je spravidla 
zosieťovanie a distribúcia záťaže úloh pomedzi uzly, čím sa zvyšuje oneskorenie a závislosť na externých 
činiteľov a od prepájacích prvkoch.

\subsection{Redundancia nižších vrstvách RM OSI}
Odolnosť sieťovej infraštruktúry proti poruchám a výpadkom prepínačov alebo smerovačov, s vedľajším
rovnako dôležitým efektom navýšenia dátového prietoku komunikačných spojov, sa utužuje protokolmi
agregácie liniek na spojovej vrstve (L2) a viaccestným smerovaním na sieťovej vrstve (L3) 

\textbf{Agregácia liniek} dostáva opodstatnenie v situáciach, kedy nestačí prenosová rýchlosť
jediného sieťového rozhrania a je nevýhodné, či dokonca nemožné vymeniť sieťovú kartu.
Vtedy sa oplatí zoskupiť niekoľko fyzických rozhraní do jedného logického linku (technika 
nazývaná ako \enquote{trunking} alebo \enquote{bonding}). Navyšovanie prenosových rýchlostí
medzi generáciami technológii na fyzickej vrstve môže takto prebiehať lineárne. Ak je vyžadovaný
GigabitEthernet (1 Gbit/s), ale nie je k dispozícii, dokáže Trunked Fast Ethernet
zabezpečiť rýchlosti 200 - 800 Mbit/s, v porovnaní s obyčajným 100 Mbit/s Fast Ethernet. 
Rámce sú striedavo posielané aktívnymi redundantnými linkami vrámci zhluku, čím
sa zabezpečí zvýšená dostupnosť a symetrizácia záťaže na participujúcich portoch.

Link Aggregation Control Protocol (LACP), ktorý je súčasťou štandardu IEEE 802.3ad, dovoľuje sieťovému 
zariadeniu vyjednanie automatické združenia liniek výmenou LACP paketov medzi partnermi.
Priebežnými keepalive správami kontroluje LACP priechodnosť spoju pre zamedzenie straty paketov
poslaním do nefunkčnej linky a overuje chyby spôsobené nesprávnym fyzickým zapojením prevedením
\enquote{loopback links} alebo \enquote{split-trunk} vznikajúci prekrížením kabeláže. V Mikrotik
router OS sa pre zhluk liniek pomenuje nové zdieľané logické rozhranie, ktorému sa následne
priradí IP adresa\footnote{\url{https://help.mikrotik.com/docs/display/ROS/Bonding}}.
V Cisco IOS sa najprv nastaví kanál do ktorého sú pridané porty\footnote{\url{https://www.cisco.com/c/en/us/td/docs/ios/12_2sb/feature/guide/gigeth.html}}.
\begin{lstlisting}
# Mikrotik - určenie rozhraní na agregáciu liniek s LACP a zdieľanie IP adresy
interface bonding add name=trunk1 mode=802.3ad slaves=ether1,ether2
ip address add address=192.168.0.1/24 interface=trunk1
# Cisco - pridanie rozhraní do skupiny kanálov č.1 so zvolenou IP adresou
interface port-channel 1
   ip address 192.168.0.1 255.255.255.0
interface range g2/0/0-1
   no ip address
   channel-group 1 mode active
\end{lstlisting}

\textbf{Viaccestné smerovanie} pridáva spoľahlivosti tam, odkiaľ vedú do cieľovej destinácie aspoň
dve trasy. Equal-cost multi-path routing (RFC 2991) je stratégia smerovania, kedy sa striedavo posielajú 
pakety cez viaceré ďalšími skokmi s rovnako dobrými metrikami cesty. V praxi je tento prístup málo 
uplatňovaný pre komplikácie viažuce sa k dynamickému výberu spomedzi dostupných smerov pre rovnomerné 
rozloženie záťaže. Rozličné spojenia tiež zvyknú mať rôzne veľkosti MTU (maximum transmission unit) a 
variabilné oneskorenia vedúce k zbytočnému preusporiadaniu paketov správy mimo poradia. Pri použití mnohých
alternatívnych preskakujúcich ciest dochádza k strate paketov. 

Existujú však postupy, ktoré majú zmierniť popísané negatíva výberu pre next-hop, ale vyžadujú udržovanie si 
stavu prebiehajúcich tokov alebo zvýšené výpočtové nároky pre voľbu next-hop. \emph{Modulo-N Hash} 
presmeruje paket na cestu podľa identifikátora toku z hlavičky paketu (najčastejšie zdrojová a cieľová 
adresa) modulo počtu dostupných uvažovaných skokov. Ak dôjde k zmene musí sa upraviť $(N-1)/N$ tokov
 \cite{RFC2991}. \emph{Hash-Threshold} rovnomerne mapuje uzly do výstupu hašovacej funkcie a podľa 
porovnania hašu identifikátoru toku s hranicami oblasti je zvolený next-hop. Pri zmenách sa upravuje cesta 
štvrtine až polovici tokov. \emph{Highest Random Weight} počíta hash zakaždým zároveň s hlavičky paketu a
kľúča pre next-hop a zvolí next-hop s najvyšším výsledným číslom. Za väčšej časovej náročnosti
mení pri pridaní alebo odobratí cesty smer už len $1/N$ tokov. Cisco router využívajúci na smerovanie
OSPFv2 uplatní ECMP jednoduchým nastavením\footnote{\url{https://www.techrepublic.com/article/how-to-configure-equal-cost-multi-path-in-ospf/}}:
\begin{lstlisting}
router ospf 1
maximum-paths 2
\end{lstlisting}

\subsection{Vysoká dostupnosť klastra}
%%%%%
% RFC2338
% Opísať rozdiel medzi základnou dostupnosťou a vysokou dostupnosťou – zdroj 1
% Protokol VRRP – mechanizmus fungovania, virtálna ip adresa – paket traces
% active-active/active-passive
% Keepalived konfigurácia dvoch Linux zariadení s failover
%3.Netlink Interface: Sets and removes VRRP virtual IPs on network interfaces.
%4.Multicast: VRRP advertisements are sent to the reserved VRRP MULTICAST group (224.0.0.18).
%\cite{keepalived-docs}
% SMTP monitorovanie

% IGMPv3 224.0.0.18 Join / Leave Group
% ARP Gratious
% VRRP Anouncement

\begin{lstlisting}
vrrp_instance malina {          # VRRP inštancia
    state MASTER|BACKUP         # štandardný stav inštancie
    interface eth0              # sieťové rozhranie
    virtual_router_id 1         # VRRP router id pre inštanciu
    priority 100|90             # priorita VRRP routera
    advert_int 1                # advertisement interval v sekundách
    virtual_ipaddress { 192.168.0.50 } # VRRP virtuálna IP adresa
}
\end{lstlisting}

\subsection{Reverzné proxy na vyvažovanie záťaže}
% DNS A záznamy (BIND config)
% Reverzná proxy – pool serverov
\subsubsection{HAProxy}

\subsubsection{Nginx}

\subsection{Algoritmy vyvažovania záťaže}
% Load balancing algoritm comparaison (http://www.linuxvirtualserver.org/docs/scheduling.html)
% Round Robin
% Weighted Round Robin 
% Least Connection
% Weighted Least Connection
% Destination Hashing
% Source Hashing


\section{Monitorovanie webovej aplikácie}

\subsection{HTTP hlavičky}

\subsection{Metriky}
% Syslog a Common Logfile Format

\subsection{Zabbix}

\subsection{Simulácie útokov a záťažové testy}
% Popis skriptu cez raw sockety
% Docker-compose: útoky ---> Haproxy/Nginx ---> 1-4x Apache Web site

% Typy útokov (UDP flood, SYN Flood, Slow Loris so spoofed IP)
% Meniť počet horizontálych inštancií
% Meniť počet workerov

% Pridaj do zabbixu ostatné hosty
% Monitoring na 
%	- Apache (Access Log, Error log)
%	- Haproxy (HAlog, https://www.haproxy.com/blog/introduction-to-haproxy-logging/)
%	- NGINX z logov (https://docs.nginx.com/nginx/admin-guide/monitoring/logging/)
%Sleduj network traffic na eth0 pri záplave:  vnstat, zabbix
%Sleduj počet spojení pri load balancingu

%\begin{lstlisting}
%ab -n 100 -c 10 http://192.168.100.82/
%hping 192.168.0.2 --udp -p 80 --flood  
%hping 192.168.0.2 --syn -p 80 --spoof 192.168.1.1

%slowhttptest -c 1000 -H -g -o my_header_stats -i 10 -r 200 -t GET -u https://192.168.100.82/ -x 24 -p 10
%thc-ssl-dos 192.168.100.82 80 --accept
%slowhttptest -u http://192.168.0.2/ -c 100

%wireshark
%\end{lstlisting}

\printbibliography[title={Literatúra}]

\end{document}
