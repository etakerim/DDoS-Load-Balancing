\documentclass[11pt, a4paper]{article}

\usepackage[slovak,english]{babel}
\usepackage[utf8]{inputenc}
\usepackage[T1]{fontenc}
\usepackage{geometry}
\usepackage{hyperref}
\usepackage{csquotes}

\usepackage[style=iso-numeric,backend=biber]{biblatex}
\addbibresource{references.bib}
\AtBeginBibliography{\small}

\geometry{
	a4paper,
	margin=2.5cm,
	top=2cm,
	bottom=2cm
}

\begin{document}
\begin{titlepage}
    \hspace{0pt}
    \centering
    \vfill
    \large Princípy informačnej bezpečnosti \\
    \vspace{0.4cm}
    \Large Špecifikácia projektu \\
    \vspace{1cm}
    \Large \textbf{Zvýšenie odolnosti webových aplikácií proti útokom typu DDoS, 
          \\za pomoci horizontálneho škálovania  \\}
    \vspace{2.5cm}
    \normalsize Miroslav Hájek \\[0.2cm]
	Akademický rok: 2020 / 2021 \\[0.1cm]
	Fakulta informatiky a informačných technológií, \\
	Slovenská technická univerzita v Bratislave
    \vfill
\end{titlepage}

%\pagenumbering{gobble}
%\tableofcontents
\pagenumbering{arabic}
\setcounter{page}{1}

\section*{Vymedzenie predmetnej domény}

\subsubsection*{Útoky odmietnutia služby}
Pozornosť projektu bude sústredená hlavne na aspekt dostupnosti ako jeden z kľúčových prvkov 
informačnej bezpečnosti. Pozrieme sa na hrozbu útokov typu odmietnutia služby (Denial of Service) zadelených 
podľa vektorov útoku. Určite neprehliadneme ani ich distribuovanú realizáciu prostredníctvom 
botnetov, nakoľko oplývajú potenciálne významnou ničivou silou v kybernetickom priestore. Preskúmame 
stratégie predchádzania voči vyskytujúcim sa rizikám príkladmi vhodnej konfigurácie infraštruktúry. Následne 
objasníme spôsoby odhalenia útokov monitorovaním kľúčových metrík cez Zabbix a dekompozíciou záznamov z 
logov. 

\subsubsection*{Ochrana a redundancia aplikačných serverov}
Popísaná prevencia bude spočívať v predstavení odlišných techník obrany proti škodlivo nadmernej sieťovej 
premávke na rôznych vrstvách sieťového modelu RM OSI. Pre úplnosť letmo spomenieme tiež ochranu aktívnymi 
sieťovými prvkami. Prioritne sa však budeme venovať vyvažovaniu záťaže (Load balancing) medzi viaceré redundantné servery s horizontálne škálovanou webovou aplikáciou spojeného s využitím rate limiting. Pričom cieľom bude dosiahnuť vyššiu spoľahlivosť a odolnosť proti výpadkom (Failover). 

Na aplikačnej vrstve porovnáme starší prístup vyvažovania záťaže cez DNS záznamy a súčasné obľúbené 
open-source softvérové riešenia reverznej proxy a load balancera: HAProxy a NGINX. Pri simultánnej prevádzke 
viacerých inštancií aplikácie vzniká problém so zachovaním perzistentných relácií (s cookies). V náväznosti na to, popíšeme rozdiely medzi dostupnými algoritmami vyvažovania záťaže a v situáciach s bezstavovým a stavovým spojením.

\subsubsection*{Riadenie prístupu na stroje za reverznou proxy}
Okrem pohľadu na nasadenú webovú aplikáciou predstavujúcu dopytovanú službu ako celok je nutné
brať na zreteľ ochranu jednotlivých serverov, ktoré využívame na zabezpečenie kontinuálnej dostupnosti.
Pozrieme sa na riadenie administrátorského prístupu cez ACL podporeného voľbou 
silných hesiel a znemožnením ich skúšania útočníkom alebo zavedením TLS. Zároveň je potrebné poohliadať
sa na nebezpečie plynúce z prílišnej transparentnosti ohľadom používaných systémov, ktoré môžu vynášať hlavičky protokolu HTTP a uľahčiť tým nájdenie zraniteľností hackerom.

\section*{Tématické rozčlenenie problematiky}
Zo zamerania vyplývajú nasledujúce tématické sekcie, s ktorými je asociovaná príslušná nosná terminológia, techniky, nástroje, perspektívne na prejednanie v projekte:

\subsubsection*{1. Dostupnosť ako bezpečnostný atribút}
Význam elementu dostupnosti informácií ako piliera 
informačnej bezpečnosti (CIA triáda). Motivácia a príležitosti na kybernetický útok DDoS z kriminalistického hľadiska. Objasnenie výberu často vytipovaných obetí. Zdroje: \cite{availability}, \cite{why-attack}.

\subsubsection*{2. Anatómia útokov Denial of Service}
Najčastejšie cielené vektory útoku pre hrozbu odmietnutia 
služby: zahltenie linky, zaplnenie stavových tabuliek, vyčerpanie výpočtových zdrojov. Formy realizácie 
najmä v podobe botnetov ovládaných cez C\&C. Taxonómia DDoS útokov rozlíšených do troch hlavných 
kategórii s analýzou konkrétnych zraniteľností: veľkoobjemové (volumetrické) útoky s 
podskupinou amplifikačných reflexívnych útokov, protokolové a aplikačné útoky. Početnosť výskytov takýchto 
hrozieb v nedávnom období a rola aktívnych sieťových zariadení v usmernení nadmernej premávky: black hole filtering, IP black lists, NAT s ACL. Zdroje: \cite{ddos-anatomy-2004}, 
\cite{botnets}, \cite{radware-ddos}, \cite{enisa-ddos}, \cite{ddos-attacks}, \cite{csirt-ddos}.

\subsubsection*{3. Škálovanie webových aplikácií}
Princíp a dopad horizontálneho a vertikálneho škálovania 
aplikácie dostupnej cez počítačovú sieť. Pri vertikálnom škálovaní si budeme všímať výsledky navýšenia
počtu procesov alebo vlákien u Dispatcher-Worker modelu. Horizontálnym škálovaním vytvoríme
symetrizáciou zátaže pomedzi viaceré inštancie. Podľa schopností vrstiev sieťového modelu rozlišujeme 
agregáciu liniek (\emph{bonding, trunking}) na spojovej vrstve (L2), viaccestné smerovanie protokolom ECMP 
(\emph{Equal-cost multi-path routing})  na sieťovej vrstve (L3) a reverznú proxy na nadviazanie spojení na 
transportnej (L4 - protokol TCP) a aplikačnej vrstve (L7 - protokol HTTP). Doplníme 
prístupom riešiacim load balancing DNS záznamami (nástroje: \emph{dnsmasq}, 
\emph{bind9}). Zdroje: \cite{link-trunking}, \cite{RFC2991}, \cite{RFC2992}, \cite{server-load-balancing}, 
\cite{load-balance-in-distributed-system}.

\subsubsection*{4. Algoritmy vyvažovania záťaže}
Požiadavky prichádzajúce na proxy server musia byť v 
distribuovanom prostredí pridelené na práve jeden cieľový webový server, ktorý ich vybaví a vráti odpoveď.
Podľa nárokov na výkon a správanie sa jednotlivých komponentov systému existujú viaceré stratégie výberu 
destinácie. Rozlišujeme medzi technikami statického a dynamického vyvažovania záťaže. Medzi statické 
algoritmy patria \emph{náhodné pridelenie}, \emph{round-robin}, \emph{najmenej pripojení} a ich vážené 
varianty, \emph{konzistentné hašovanie} IP adresy zdroja alebo URI cieľa. Dynamické stratégie pridávajú 
komplexnosť v tom, že uzol rozdeľujúci požiadavky musí pravidelne zbierať údaje o celkovom stave ním 
spravovaných serverov. Majorita webstránok musí uchovávať stav v cookies a load balancer sa musí tomu podriadiť: riešením sú tzv. \emph{sticky sessions}. Zdroje: \cite{load-balance-algo}, 
\cite{algorithm-performance}, \cite{sticky-session}.

\subsubsection*{5. Softvérové nástroje na vyvažovanie záťaže}
Porovnáme funkčné, výkonnostné a bezpečnostné 
možnosti dvoch populárnych load balancerov HAProxy a NGINX. Predvedieme ukážky ich základnej konfigurácie
a efekt rôznych algoritmov na vyvažovanie horizontálne škálovanej aplikácie, ktorá využíva webový server NGINX. Pôjde jednak o jednoduchú vlastnú webstránku galérie so statickým obsahom, ktorá bude
v ďalšom porovnávaní rozšírená o dynamickú časť v jazyku PHP (cez FPM), s prihlasovaním na personalizovaný výber fotografií na titulnej stránke. Pri nastavovaní load balancerov proti najčastejším DDoS útokom sa tiež oprieme o odporúčania od tvorcov týchto nástrojov na prevenciu spoliehajúcu sa predovšetkým na limitovaní počtu aktívnych spojení alebo frekvencie požiadaviek na jedného klienta cez algoritmus \emph{leaky bucket} a podobne. Zdroje: \cite{nginx-http-balancer}, \cite{nginx-cookbook}, 
\cite{haproxy-docs}, \cite{haproxy-ddos}, \cite{haproxy-ddos-L7}, \cite{haproxy-php-fpm}, 
\cite{nginx-ddos-protection}, \cite{haproxy-rate-limiting}.

\subsubsection*{6. Vysoká dostupnosť služieb}
Nasadením load balancera síce dosiahneme redundanciu pre webový 
server, ale vyrobili sme centrálny bod zlyhania, ktorý eliminujeme duplikáciou \emph{HAProxy} do klastra
na dvoch strojoch v konfigurácii Active-Passive nástrojom \emph{Keepalived}. Dosiahne toho priradením jednej 
virtuálnej IP adresy pre klaster a prepnutím na zálohu (Failover) po signalizovaní BFD (Bidirectional 
Forwarding Detection) prostredníctvom protokolu VRRP (Virtual Router Redundancy Protocol). Zdroje: 
\cite{keepalived-docs}.

\subsubsection*{7. Ochrana webových serverov}
Protokol HTTP dokáže v hlavičkách odpovedí (Forwarded, Server, 
atď.) odkryť vlastnosti používaných systémov a dokonca adresy serverov skrytých za reverznou proxy. Tým 
môže odhaliť útočníkovi priamy prístup na jednotlivé servery alebo koncové body, ak nie sú chránené napr. 
ACL. Zabezpečenie spojenia s aplikáciou s TLS vložením X.509 certifikátov a proces prihlasovania do 
systému cez HTTP(S). Zdroje: \cite{csirt-hardening}, \cite{csirt-web-checklist}, \cite{RFC2616}, 
\cite{http-headers}

\subsubsection*{8. Monitorovanie webovej aplikácie}
Sledovanie stavu sieťovej premávky ako sú vyťaženosť linky, 
nadviazané spojenia a ich časová distribúcia a tiež kontrolovanie utilizácie výpočtových zdrojov, ktoré 
zabezpečujú dostupnosť služby. Reaktívna analýza útokov z loggov load balancera a webového servera v Common
Logfile format. Automatizácia upozornení na prípadné hrozby v monitorovacom nástroji Zabbix integrovanom
rozšíreniami s demonštrovanými load balancermi. Preskúmanie najdôležitejších metrík, z tých ktoré sú k 
dispozícii a kontrola ich vývoja počas DoS. Zdroje: \cite{common-log}, \cite{haproxy-logging}, \cite{nginx-zabbix}, \cite{haproxy-zabbix}

\subsubsection*{9. Simulácia útokov a záťažové testy}
Výsledky záťažových testov vlastnej webovej aplikácie 
spustenej v Docker kontajneroch cez \emph{docker compose}  a na fyzických zariadeniach Raspeberry Pi (2x Pi 3B, 1x Pi 2B). Účinky existujúcich nástrojov na stress testing: \emph{ab, t50, thc-ssl-dos, 
SlowHTTPTest, High Orbit Ion Cannon}, ale aj vlastnými viacvláknovými skriptami na UDP flood, SYN flood, Slow Loris. Pokus o vydávanie sa za rôzne počítače technikou IP Spoofing. Vyhodnotenie dopadov pre rôzny počet bežiacich inštancií web servera, počet workerov a nastavenia load balancingu.

\section*{Časový plán čiastkových cieľov}

\begin{itemize}
\item 4. týždeň: Pripravená testovacia architektúra pre load balancing a vysokú dostupnosť v Docker 
kontaineroch a na Raspberry Pi, spolu so statickou web stránkou a rozpracovaným prihlasovaním.
Sprevádzkovaný Zabbix s rozšíreniami pre HAProxy a NGINX.

\item 8.týždeň: Zrealizované vybrané DoS útoky a záťažové testy zvolenými nástrojmi aj vlastnými skriptami a 
kvantifikované výsledky. Na teoretickej úrovni rozobrané nosné témy práce, okrem sekcií: škálovanie webových 
aplikácii, monitoring aplikácie a implementácia TLS a ACL pre web servery, dopad stavových spojení na load 
balancing
\end{itemize}

%\nocite{*}
\printbibliography[title={Zoznam relevantných zdrojov pre rešerš}]

\end{document}
